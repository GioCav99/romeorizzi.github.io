\voice{{\LARGE Interessi ed Aree di Ricerca}}

{\bf Ricerca Operativa.}
    Problemi di basi di cicli, di cammini minimi, di taglio minimo, e di routing.
    Problemi di selezione di portafoglio.
    Problemi di power management.

{\bf Biologia Computazionale.}
    Haplotyping di individui e di popolazioni.
    NMR peack analysis.
    Confronto ed analisi di stringhe con struttura.
    Individuazione di Motifs.
    Algoritmica di permutazioni motivata da genomica.
    Rilevamento di pathways.
    Protein design.
    Raffinamento di modelli per la lettura dei contigs.
    Problematiche di ottimizzazione in radioterapia. 

{\bf Ottimizzazione Combinatoria.}
    Grafi,
    Matroidi,
    Colorazione di archi,
    Fattorizzazione di grafi,
    Teoria e problemi di matching,
    Basi di cicli,
    Problemi di packing e covering,
    Problemi di channel assignment.

{\bf Algoritmi.}
    Algoritmi Polinomiali e Pseudopolinomiali,
    Algoritmi Approssimati,
    PTAS e FPTAS,
    Algoritmi Distribuiti,
    Algoritmi Paralleli,
    Algoritmi Randomizzati,
    Algoritmi Euristici.
    Algoritmi efficienti per il Listing e l'Enumerazione.
    Controllo dinamico discreto.
    Reti temporali e workflows.
    Giochi combinatorici legati a problematiche di model checking.

{\bf Complessit\`a Computazionale.}
    Risultati di NP-completezza,
    Buone caratterizzazioni,
    Risultati di APX-hardness ed inapprossimabilit\`a,
    Risultati di PSPACE-completezza.
    FPT e W-hardness.
    Lower bounds condizionali.

{\bf Calcolo ed Architetture Parallele.}
    Analisi delle POPS-networks.

{\bf Reti.}
    Problemi di frequency assignment e di channel assignment.
    Problemi di scheduling.\\



\voice{{\LARGE Titoli di Studio}}

\vspace{-0.8mm}
\medskip \noindent
(giugno 1986) \
{\large \bf Diploma di Maturit\`a Scientifica}\\
{\em Liceo Scientifico ``Leonardo da Vinci'', Trento.}\\

\medskip \noindent
(dicembre 1991) \
{\large {\bf Laurea in Ingegneria Elettronica}  ad indirizzo matematico fisico}\\
{\em Politecnico di Milano. Media esami: 29/30. Voto finale: 100/100 e Lode.}

Tesi di Laurea:
{\em ``Il problema dell'albero minimo di cardinalit\`a $k$.''}
Relatore:
{\em prof.~F.~Maffioli (Politecnico di Milano, Dipartimento di Elettronica).}
Aree di interesse:
{\em Ricerca Operativa, Ottimizzazione Combinatoria.}\\

\medskip \noindent
(settembre 1997) \
{\large \bf Dottore di Ricerca}\\
{\em Dottorato in Matematica Computazionale ed Informatica Matematica, IX ciclo.
Dipartimento di Ma\-te\-ma\-ti\-ca Applicata dell'Universit\`a di Padova}.

Tesi di Dottorato:
{\em ``Impaccando $T$-tagli e $T$-giunti.''}
Relatore:
{\em prof.~M.~Conforti (Universit\`a di Padova,
                Dipartimento di Ma\-te\-ma\-ti\-ca).}
Controrelatore:
{\em prof.~A.M.H.~Gerards (Istituto di Ricerca CWI, Amsterdam).}
Aree di interesse:
{\em Ricerca Operativa, Teoria dei grafi, Combinatorica.}\\

%\newpage

\voice{{\LARGE Qualifiche e impegni lavorativi attuali}}

\subvoice{Professore associato presso la
          Facolt\`a di Scienze MM.FF.NN di Verona}
{\bf dicembre 2011 -- oggi.}
Settore MAT/09 (Ricerca Operativa).

{\bf NOTA:}  ho una abilitazione da ordinario in mat/09
(conseguita nella prima tornata e valida dal 20/12/2013 al 20/12/2019)

{\bf uffici presso il dipartimento:} Presidente di Commissione Paritetica, Membro della Commissione di valutazione assegni di tutorato per corsi di Informatica e Bioinformatica, Referente di Dipartimento per le Olimpiadi dell'informatica, Referente del Dipartimento verso il coderDojo in Verona, Membro del Collegio dei Docenti del Dottorato Interateneo in Matematica, Membro del Collegio Didattico di Informatica, Membro del Collegio Didattico di Matematica, Membro del Consiglio di Corso di Tirocinio Formativo Attivo - TFA classe A042- Informatica, Membro del Consiglio del Dipartimento di Informatica.

{\bf corsi per conto del dipartimento:} in Verona,
ho tenuto i corsi
di ``Ricerca Operativa''
per la laurea triennale in Matematica Applicata L35 (2011-12, 12-13, 13-14, 14-15, 15-16, 16-17)
e di ``Algoritmi''
per la laurea magistrale in Ingegneria ed Informatica LM18+LM32 (2011-12, 12-13, 13-14, 14-15, 15-16, 16-17).
Ho attivato e coordino il corso ``Sfide di Programmazione'' (2013-14, 14-15, 15-16).
Per la magistrale di Matematica LM40, sono responsabile del corso seminariale ``Mathematics for Decisions'' (2014-15, 15-16, 16-17).
Ho tenuto corsi per il TFA (2012-13, 2014-15) e per il PAS (2013-14, 2014-15).
Sto sperimentando dei corsi tandem (offerta rivolta dall'ateneo di Verona verso gli studenti di scuola superiore) in algoritmi (2014-15, 15-16).
All'ITIS Marconi di Verona, ho condotto pi\'u cicli di interventi e lezioni in merito alle olimpiadi di informatica (dal 2012 ad oggi),
come anche un mini-corso in Programmazione Matematica in seno al PLS (2014-15). 
Per il Dottorato congiunto in Matematica Trento-Verona tengo il corso
``Mathematical Programming'' (2013-14, 14-15, 15-16).\\



\subvoice{Altre Attivit\`a}
\indent
{\bf (impegno nelle olimpiadi di informatica)}
sia a livello nazionale, dove collaboro all'allenare la squadra italiana
ed a portare avanti le progettualit\`a collegate alle iOi ed alle Oii,
sia a livello di realt\`a locali in Trento, Bolzano, Verona, Udine.
Sul tavolo delle olimpiadi posso vantare un lungo e sostanziale impegno in varie attivit\`a e progetti. 

{\bf (collaborazione con ditte)}
da autunno 2014 ho attivato delle collaborazioni con ditte
su problematiche di ottimizzazione dei processi produttivi.

{\bf (la mia attivit\`a scientifica)}
con collaborazioni sia presso il mio dipartimento in Verona, che in Italia, che all'estero su un po' tutto il globo.


\vspace{1.8mm}

\voice{{\LARGE Esperienze di lavoro}}

\subvoice{Professore associato presso la
          Facolt\`a di Ingegneria di Udine}
{\bf ottobre 2005 -- dicembre 2011.}
Settore MAT/09 (Ricerca Operativa).
Idoneit\`a ottenuta nel giugno 2003.

{\bf (corsi)} in Udine,
ho tenuto i corsi di ``Matematica II''
per la facolt\`a di architettura
(2005-06, 06-07, 07-08, 08-09),
di ``Matematica'' (MatI + MatII)
per la facolt\`a di architettura (2010-11, 11-12),
di ``Ricerca Operativa''
per la facolt\`a di ingegneria
(2005-06, 06-07, 07-08, 08-09, 09-10, 10-11, 11-12)
e di ``Ricerca Operativa''
per la facolt\`a di architettura
(2006-07, 07-08, 08-09, 09-10).
Membro del Collegio di Dottorato in Udine,
ho ivi offerto un corso in ``Complessit\`a Computazionale''
che ha accolto anche alcuni studenti stranieri.

{\bf (tesi)} Relatore di Massimiliano Cossu
(laurea in Ingegneria Gestionale Industriale,
``Modellizzazione di un sistema di trasporti di una realt\`a aziendale'')
e correlatore di Francesco Cafarelli
(laurea specialistica in Matematica,
%110+lode,
``Algoritmi efficienti per il problema del minimum test collection'',
relatore: Prof.~Franca Rinaldi)
e di Stefano Michelini (laurea specialistica in Matematica, ``Un algoritmo per il Train Marshalling Problem'', relatore: Prof.~Franca Rinaldi) %, a.a.~2009/2010.
e di due tesi esterne all'ateneo di Udine
(Luca Nardin, laurea in Informatica, Trento,% 110+lode,
``Polynomial time instances for the IKHO problem'',
relatore: Prof.~Roberto Sebastiani)
e (Elia Calderan, laurea in Matematica, Trieste,% 110+lode,
``Approcci combinatorici al contenimento di fuochi'',
relatore: Prof.~Andrea Sgarro,
secondo correlatore: Prof.~Giuseppe Lancia).

{\bf (fondi)} Ho preso parte a progetti nazionali ed internazionali.
Un progetto Italo-Francese di cui Paola Bonizzoni era responsabile per la parte italiana
mi ha consentito una visita al gruppo di Vialette in Parigi.
% Attivita' di ricerca in ambito bioinformatico, sotto un progetto italo-francese gestito da Paola Bonizzonini, mi hanno portato a Nantes e Parigi, ed ho ospitato Guillaume Blin e Florian Sikora in Udine. 
Con il gruppo di Ricerca Operativa in Udine (Paolo Serafini, Giuseppe Lancia, Franca Rinaldi, Romeo Rizzi) eravamo inseriti nel PRIN~2006 ``Modelli di data mining e di ottimizzazione per le applicazioni biologiche e mediche'' con coordinatore scientifico nazionale Carlo Vercellis, il titolo del programma dell'unit\'a di Udine, con responsabile Giuseppe Lancia,
era ``Algoritmi di ottimizzazione per l'analisi comparativa di dati genomici di grandi dimensioni''.

{\bf (orientamento)} Ho portato avanti il mio impegno nelle olimpiadi di informatica,
sia sul piano nazionale ed internazionale per mandato di AICA,
che sul piano locale in Friuli (Udine), Trentino-AltoAdige (Trento, Bolzano e Cles), Lombardia (con ACOF e con l'ITC Tosi Di Busto Arsizio), Bologna.\\



\subvoice{Ricercatore presso la
          Facolt\`a di Scienze di Povo (Trento)}
{\bf marzo 2001 -- ottobre 2005.}
Settore INF/01 (Informatica).

{\bf (corsi)} Dopo l'ottenimento del ruolo,
ho tenuto i corsi di ``Laboratorio di Algoritmi e Strutture Dati''
(II$^o$ semestre 2000-01),
``a PhD Course in Linear Programming'' (nov 2001 - gen 2002),
``a Phd course in Computational Molecular Biology'' (marzo 2002 ed aprile 2003),
``Algoritmi e Strutture Dati I'' (I$^o$ bimestre 2002-03),
``Complessit\`a Computazionale'' (III$^o$ bimestre 2002-03,
IV$^o$ bimestre 2003-04 e II$^o$ semestre 2004-05).
Fuori sede, nel II$^o$ semestre 2004-05
ho tenuto un modulo in Programmazione Lineare
titolato ``Tecniche Anvanzate per la soluzione di problemi di ottimizzazione
combinatorica''
presso il Dipartimento di Matematica e Informatica
dell'Universit\`a di Perugia.

{\bf (fondi)} Ho preso parte a progetti locali, nazionali, ed internazionali.
Presentando un progetto congiunto con Fertin (Nantes) e Vialette (Parigi),
ho ottenuto un fondo Galileo dell'Universit\`a Italo-Francese,
di cui nel 2005 sono stato responsabile
per la parte Italiana (Trento, Milano-Bicocca e Udine).
Con il gruppo di Algoritmi in Trento (Alan Bertossi, Cristina Pinotti, Romeo Rizzi)
eravamo unit\'a nel progetto di ricerca ``Algorithms for wireless networks''
REAL-WINE 2001--03 di cui Bertossi era anche responsabile nazionale.
Progetto WILMA, ed altri progetti della provincia di Trento fuori e dentro il contenitore CreateNet. Partecipazione con il gruppo di reti (Roberto Battiti, Mauro Brunato, Romeo Rizzi).

{\bf (orientamento)} Ho tenuto corsi ed altre iniziative
per la preparazione alle Olimpiadi di Informatica
sia per studenti delle Scuole Superiori di Trento
che di Bolzano
(inverni 2001-02, 2002-03, 2003-04, 2004-05, 2005-06, 2006-07 e 2007-08).
In particolare, nel 2004 ho collaborato utilmente
per portare a Trento la fase nazionale delle iOi
e sono stato arruolato dal
Comitato Olimpico dell'AICA come allenatore e selezionatore
per la squadra nazionale;
ruolo che ho ricoperto fino ad oggi,
collaborando alla buona riuscita delle fasi nazionali
in Taormina 2005, Milano 2006, Bari 2007, Desenzano 2012
e con le settimane di allenamenti (Tirrenia 2004,
Pisa 2005-06-07, Volterra 2008-2012, Desenzano 2010-12),
ed accompagnando la squadra italiana alle olimpiadi
in Polonia (2005) come team leader a fianco di Roberto Grossi.

{\bf (tesi di laurea)} Relatore di Marco Rospocher
(vecchia laurea in Matematica,% 110+lode,
 ``All-Pairs and Matching-like Shortest Paths Algorithms''),
di Michele Vescovi
(laurea triennale in Informatica,% 110+lode,
 ``Algoritmi per la segmentazione audio basati
   sul criterio di informazione bayesiano''.
 Responsabile esterno: Mauro Cettolo - IRST,
gli algoritmi di segmentazione audio codificati da Michele
nel suo periodo di interinato presso l'IRST
e nel suo conseguente percorso di tesi di laurea sono stati
acquisiti dalla RAI),
e di Paolo Zotti
(laurea triennale in Informatica,
 tesi compilativa: ``Gli algoritmi di ordinamento'').

{\bf (stagisti)} Ho seguito, in qualit\`a di Advisor,
i seguenti stagisti: Nitin Saxena~\footnote{uno degli autori di ``PRIMES is in P''} e Shivi S. Bansal (primavera 2001),
Shashank Ramaprasad e Shashanka Madhusudana (primavera 2003).

{\bf (tesi di dottorato)} Nel periodo 2002-05 sono stato l'Advisor per il Dottorato di Marco
Rospocher presso il DIT (tesi: ``On the computational complexity
of enumerating certificates of $NP$ problems'').\\


\subvoice{Ricercatore R1 presso l'I.R.S.T.}
{\bf Agosto 2000 -- febbraio 2001:}
inserito nel gruppo CBR (Case Based Reasoning, coordinato da Paolo Avesani)
della divisione SRA
(Sistemi di Ragionamento Automatico, diretta da Paolo Traverso) in IRST.
L'IRST (Istituto Ricerca Scientifica e Tecnologica)
\`e un organo dell'ITC (Istituto Trentino Cultura).\\

\subvoice{Posizioni temporanee presso Universit\`a ed Enti di Ricerca all'estero}
{\bf Agosto 99 -- ottobre 99:}
     Assistant Research Professor
     al  BRICS (Universit\`a di Aarhus, Denmark). 

\noindent
{\bf Aprile 2000 -- giugno 2000},
{\bf novembre 99 -- dicembre 99},
{\bf aprile 99 -- giugno 99},
{\bf novembre 98 -- dicembre 98:}
      Ho ricoperto, per un totale di 10 mesi,
      una posizione di ricerca su fondi DONET
      presso il centro di ricerca CWI di Amsterdam.
      Ero inserito nel gruppo PNA (Probability, Networks, Algorithms)
      sotto la guida dei professori Alexander Schrijver
      e Bert Gerards.\\
%mio primo periodo al CWI di Amsterdam, nel gruppo PNA (Probability, Networks, Algorithms) guidato dei professori Alexander Schrijver e Bert Gerards.
%Posizione di ricerca  su borsa Europea (fondi DONET).

\subvoice{Borsista post-dottorato}
{\bf Giugno 98 -- giugno 99:}
Per borsa
bandita dall'Universit\`a di Padova
ed usufruita
presso il Dipartimento di Matematica 
della stessa,
sotto la guida del prof.~Michele Conforti.\\


\subvoice{Contratti e Collaborazioni}
{\bf 2004--oggi:} arruolato dal Comitato Olimpico dell'AICA
come allenatore e selezionatore
della nazionale italiana per le edizioni 2004--15
delle {\em Olimpiadi di Informatica}.

\noindent
{\bf 2001--oggi:}
contratti con il {\em Liceo Scientifico Galileo Galilei}
per dei corsi di preparazione alle {\em Olimpiadi di Informatica}
e rivolti agli studenti interessati
di tutte le scuole superiori della provincia di Trento.
In seno all'attivit\`a di preparazione alle olimpiadi
si inseriscono anche,
partendo dal 2002, contratti
sia con l'{\em I.T.I.S.
Max Valier} di Bolzano che
con la {\em Sovrintendenza Scolastica Tedesca} di Bolzano
per docenze presso l'I.T.I.S.
Max Valier,
e nel periodo 2006-08 dei contratti presso
l'{\em Istituto di Istruzione Bertrand Russel} di Cles.
%
%e per un corso intensivo di preparazione
%alla fase nazionale, tenuto per conto del {\em DIT} presso il {\em Dipartimento
%di Informatica e Telecomunicazioni} stesso in Povo (febbraio 2003),
%e rivolto a studenti sia del Trentino che dell'Alto Adige.

\noindent
{\bf 2009--10:}
contratti con ACOF (Associazione Culturale Olga Fiorini)
per organizzazione e docenza
in Learning Weeks rivolte a ragazzi da scuole della Lombardia,
da bandi su fondi Europei,
e coordinate/gestite/organizzate anche dall'ITC Tosi di Busto Arsizio.
% nel periodo 2009-2010, in Lombardia riuscimmo ad organizzare diverse Learning Week su fondi europei, rivolte a ragazzi di scuole superiori della Lombardia, proponendo percorsi di eccellenza in ambito informatico/algoritmico. Io ero il docente principale di riferimento ed il coordinatore ed organizzatore del corso. Per alcune di queste settimane l'organizzazione e la cura del bando fu sostenuta dall'ITC Tosi di Busto Arsizio. Per altre da ACOF (Associazione Culturale Olga Fiorini). Con queste istituzioni io poi avevo regolari contratti.


\noindent
{\bf 2010:}
contratto per una settimana di studio intensiva residenziale in Volterra
con una classe del Liceo Scientifico Enrico Fermi di Bologna. 

\noindent
{\bf 2012:}
contratto per una settimana di studio intensiva
con una classe di ragazzi del Liceo Scientifico Enrico Fermi di Bologna. 

\noindent
{\bf Febbraio -- luglio 2000:}
contratto con la {\em Libera Universit\`a di Bolzano}
in merito al {\em Progetto Giornalino Virtuale}.
Questo progetto dell'Univerit\`a di Bolzano
ha coinvolto i bambini di varie scuole elementari
nella realizzazione di un loro giornalino in rete.
Il progetto era sorto
come momento di ricerca sperimentale ed applicata
in pedagogia dell'infanzia, con l'obbiettivo
di esplorare le opportunit\`a offerte dalle nuove tecnologie. 
Ho partecipato a questo progetto
in qualit\`a di tecnico
per le problematiche di tipo informatico,
come realizzatore delle pagine centrali del sito,
come coordinatore verso le pagine della redazione (gestite da bambini
e maestri) e come docente nei corsi di formazione per
maestri delle scuole elementari coinvolte.
(La quasi totalit\`a delle scuole italiane in provincia di Bolzano,
ed anche qualche scuola tedesca).

\noindent
{\bf Giugno 97 -- aprile 98:}
contratto con il gruppo LEA {\em (Laboratory for Experimental Algorithmics)}
per la realizzazione di moduli software
presso il {\em Dipartimento di Matematica dell'Universit\`a di Trento}.
Uno dei temi del gruppo LEA
era la realizzazione di ``intertools''
disponibili in rete per la soluzione tramite euristiche
di problemi NP-completi.
In seno a tale progetto,
il sottoscritto era responsabile per lo sviluppo dell'intertool
per il ``graph partitioning''.\\

\subvoice{Docenze universitarie (da esterno)}
{\bf Secondo semestre anno accademico 2004-05:}
professore a contratto per un modulo in Programmazione Lineare
titolato ``Tecniche Anvanzate per la soluzione di problemi di ottimizzazione
combinatorica''
presso il Dipartimento di Matematica e Informatica
dell'Universit\`a di Perugia.\\

\subvoice{Docenze universitarie (ante ruolo)}
{\bf Secondo semestre anno accademico 97/98:}
professore a contratto per 
un corso integrativo di Programmazione Combinatoria
nell'ambito del corso di Programmazione Matematica
al Dipartimento di Matematica dell'Universit\`a di Trento.\\

\subvoice{Esercitazioni al Diploma di Laurea (ante ruolo)}
{\bf Secondo semestre anno accademico 96/97:}
esercitatore del corso di Analisi II
per il Diploma di Ingegneria Informatica ed Automatica a Rovereto.\\

\subvoice{Attivit\`a come ingegnere e Progetti}
{\bf Gennaio 94 -- dicembre 96:}
iscritto all'Albo degli Ingegneri di Trento,
ho presentato un progetto edile
per un'abitazione in Vezzano (Trento).
Tale progetto \`e stato approvato e realizzato.\\


\subvoice{Supplenze presso istituti di scuola superiore}
%Ho svolto diversi periodi di supplenze temporanee
%presso vari istituti superiori, sia prima che dopo la laurea
%(prima della laurea solo corsi serali).
%Tra questi, due incarichi annuali:
%
%\noindent
%{\bf anno scolastico 97/98:}
%   supplenza in ``Matematica'' ed in ``Matematica ed Informatica''
%   all' {{\sc I.P.C.} ``L.~Battisti''}  di Trento.
%
%\noindent
%{\bf anno scolastico 89/90:}
%supplenza in ``Elettrotecnica'' ed in ``Misure Elettriche''
%all' {{\sc I.T.I.S.} ``P.Hensenberger''}
%di Monza.
%
%\noindent
%In totale,
%ho svolto i seguenti periodi di supplenza:
Ho tenuto le seguenti supplenze
presso istituti di scuola superiore:\\

\begin{center}
\begin{tabular}[c]{||c|p{0.80in}|p{1.25in}|p{1.25in}|p{1.25in}||}
 \hline \hline
  anno      & periodo & scuola & discipline di insegnamento & note\\
 \hline \hline
  89-90     & {\bf intero anno scolastico} & I.T.I.S. Hensenberger (Monza)
            & (elettrotecnica) (misure elettriche)
            & prima della laurea (e solo serale) \\
% dal 14-11-89 al 31-8-90
 \hline \hline
  92-93     & dal 21/9/92 al 17/10/92
            & I.T.I. Marconi (Rovereto)
            & (informatica industriale) % codice: LIV 
              (matematica applicata) % codice: LXIV
            & nessuna \\
  92-93     & dal 26/10/92 al 14/11/92 & I.T.C. Martini (Mezzolombardo)
            & 038A (fisica) & nessuna \\
  92-93     & dal 15/11/92 al 10/06/93 & I.T.C. Martini (Mezzolombardo)
            & 038A (fisica)
            & nomina valida ai soli fini giuridici (servizio militare) \\
 \hline \hline
  93-94     & dal 13/10/93 al 18/11/93 & I.T.I.S. Buonarroti (Trento)
            & 035A (elettrotecnica e applicazioni) & nessuna \\
  93-94     & dal 12/2/94 al 26/2/94 & I.T.C. Martini (Mezzolombardo)
            & 048A (matematica applicata) & nessuna \\
 \hline \hline
  95-96     & dal 22/9/95 al 6/11/95 & I.T.I.S. Buonarroti (Trento)
            & 035A (elettrotecnica e applicazioni) & 1 giorno di
  assenza per concorsi \\
 \hline \hline
  96-97   & dal 17/4/97 al 21/4/97 & I.P.C. Don Milani (Rovereto)
            & 042A (informatica) & nessuna \\
 \hline \hline
  97-98     & {\bf intero anno scolastico} & I.P.C. Battisti (Trento)
            & 047A (matematica) (matematica ed informatica)
            & nessuna \\
% dal 15-9-97 al 30-6-98 con proroga al 31-8-98
 \hline \hline
  98-99   & dal 17/9/98 all' 1/10/98 & I.T.C.G. Floriani (Riva)
            & 048A (matematica applicata) & nessuna \\
  98-99   & dal 11/1/99 al 11/1/99 & I.T.C.G. Fontana (Rovereto)
            & 047A (matematica) & nessuna \\
 \hline \hline
  99-2000   & dal 15/1/00 al 31/3/00 & I.T.I.S. Buonarroti (Trento)
            & 047A (matematica) & nessuna \\
 \hline \hline
\end{tabular}
\end{center}


%\vspace{1.8mm}

\voice{{\LARGE Abilitazioni}}
\begin{itemize}
\vspace{-4.0mm}
   \item[]
      {\em Accademia} \ 
       - abilitazione da ordinario in mat/09
(conseguita nella prima tornata e valida dal 20/12/2013 al 20/12/2019)

      {\em Professione di Ingegnere} \ 
       - esame di stato: Milano, giugno 1992.
                    
   \item[]
      {\em Insegnamento matematica per le superiori (047A)} \ 
       - concorso ordinario: Bolzano, marzo 2000.

   \item[]
      {\em Insegnamento fisica per le superiori (048A)} \ 
       - concorso ordinario: Bolzano, maggio 2000.
\end{itemize}

\voice{{\LARGE Servizio militare}}
\begin{itemize}
\vspace{-4.0mm}
   \item[] {\bf Assolto}: \ \ Incorporato il 16 novembre 92.
                          \ Congedato il 15 novembre 93.
\end{itemize}


%\voice{{\LARGE Lingue straniere}}
%\begin{itemize}
%\vspace{-4.0mm}
%   \item[] {\bf Inglese:}  Buona conoscenza orale e scritta.
%                    Ho frequentato un
%                    corso Advanced Level di durata trimestrale 
%                    in Cambridge. A conclusione del corso ho
%                    conseguito un diploma.
%                    
%   \item[] {\bf Tedesco:}  Prima lingua straniera
%                    dalla 4$^a$ elementare
%                    alla 5$^a$ superiore.  % Conoscenza scolastica.
%
%   \item[] {\bf Francese:} Conoscenza di base, maturata
%                    a seguito di periodi di permanenza
%                    in Grenoble.
%\end{itemize}
%
%
%\voice{{\LARGE Pratica nelle tecnologie informatiche}}
%\begin{itemize}
%\vspace{-4.0mm}
%   \item[] {\bf Sistemi operativi:}
%          MS-DOS, Concurrent DOS, Unix, Linux, Windows.
%   \item[] {\bf Linguaggi:}
%          Ho lavorato in: C++, C, Pascal, Fortran, Modula2, BASIC, Clipper.
%   \item[] {\bf Reti:}
%          Ho collaborato alla realizzazione
%          di diversi siti web, con la scrittura di pagine HTML e la realizzazione
%          di software CGI. 
%   \item[] {\bf Attestati:}
%          Diploma di Operatore Meccanografico.
%          Conseguito nel periodo di leva presso il
%          P.A.C. di Bolzano.
%\end{itemize}



\voice{{\LARGE Altri periodi all'estero per studio e ricerca}}
   \begin{itemize}
\vspace{-4.0mm}
%      \item[] {\bf Giugno -- Agosto 1994:}
%            Cambridge (Inghilterra).
%            Per approfondire la mia conoscenza della lingua inglese.  
      \item[] {\bf Novembre 1995, Ottobre 1996:} 
            Ospite
            del prof.~Andr\'{a}s Seb\"o
            presso i Laboratoires
            IMAG % Informatique Mathematiques Appliquees Grenoble
            e Leibniz
            dell'Universit\`a di Grenoble, Francia.
      \item[] {\bf Novembre--Dicembre 2000, Gennaio--Febbraio 2003:} 
            Ospite
            del prof.~Pavol Hell
            presso il Dipartimento di Matematica
            della Simon Fraeser University (SFU)
            di Vancouver, Canada;
            del prof.~Gary MacGillivray
            presso il Dipartimento di Matematica
            della University of Victoria (UV), Canada
            e del prof.~Rick Brewster
            presso il Dipartimento di Computer Science
            della University of Scherbrook (Montreal), Canada.
      \item[] {\bf Agosto 2001:}
            Ospite del BRICS (Universit\`a di Aarhus, Denmark).
      \item[] {\bf Novembre--Dicembre 2004:} 
            Ospite
            del prof.~Pablo Moscato
            presso il Bioinformatics Center
            dell'University of Newcastle, Australia.
            Visita anche all'Australian National University in Canberra.
% scritto su domanda: Research Professor presso il Bioinformatics Center dell'University of Newcastle, Australia. Fondi in parte dell'Universita' di Newcastle, in parte da progetti del Prof. Pablo Moscato, ed in parte dell'Australian National Council. (dal 3 novembre al 22 dicembre). 
      \item[] {\bf Settembre--Ottobre 2005:} 
            Ospite
            del prof.~St\'ephane Vialette
            presso l'Universit\'e Paris-Sud (Orsay).
      \item[] {\bf Dicembre 2005:} 
            Ospite
            del prof.~Guillaume Fertin
            presso l'Universit\'e Nantes.
      \item[] {\bf Novembre 2009:}
            Invited Professor (“Professor Invitee”) 
            presso l'Universit\'e Paris-Est - Marne-la-Vall\'ee.
            Ospite del prof.~St\'ephane Vialette.
            Contratto di un mese a scopo di ricerca presso il gruppo del Prof.~St\'ephane Vialette.
      \item[] {\bf Febbraio 2013:} 
            Invited Professor (“Professor Invitee”) 
            presso l'Universit\'e Paris-Est - Marne-la-Vall\'ee.
            Ospite del prof.~St\'ephane Vialette.
            Ospite
            del prof.~St\'ephane Vialette
            presso l'Universit\'e Paris-Est - Marne-la-Vall\'ee.
            Contratto di un mese a scopo di ricerca presso il gruppo del Prof.~St\'ephane Vialette.
      \item[] {\bf Novembre 2015:} 
            Invited Professor (“Professor Invitee”) 
            presso l'Universit\'e Paris-Est - Marne-la-Vall\'ee.
            Ospite del prof.~St\'ephane Vialette.
            Contratto di un mese a scopo di ricerca presso il gruppo del Prof.~St\'ephane Vialette.
   \end{itemize}




\voice{{\LARGE Seminari}}

Ho divulgato
i risultati dei miei lavori di ricerca mediante
seminari presso i seguenti istituti (lista non pi\`u aggiornata):
Laboratorio IMAG del CNRS in Grenoble (1995),
Laboratorio Leibniz dell'Universit\`a di Grenoble (1996, 2003),
Istituto IASI del C.N.R. in Roma (1997, 1999),
DEIS dell'Universit\`a di Bologna (1997, 1999, 2001),
DSI dell'Universit\`a di Bologna (2008),
Istituto di Ricerca CWI in Amsterdam (1998),
DEIS del Politecnico di Milano (2000, 2003),
Dip.~Matematica ed Informatica Universit\`a di Udine (2006),
Dipartimento di Informatica della Bicocca di Milano (2003, 2006, 2008, 2016),
LIGM - Université Paris-Est Marne-la-Vallée (2009,2013,2015)
Dip.~Informatica Universit\`a di Verona (2008),
DSMI dell'Universit\`a di Reggio Emilia (2000),
Istituto IRST dell'ITC di Trento (2000, 2001),
Math. Dept. della Simon Freaser University di Vancouver (2000, 2003),
Math. Dept. della University of Victoria (2003),
Dipartimento di Elettronica del Politecnico di Torino (2003),
Engineering Dept. dell'Universit\`a di New Castle (2004).\\


%\newpage

\voice{{\LARGE Riviste Internazionali}}

\begin{etaremune}
  \vspace{-3.0mm}
  
  \item {\sc Enrico Fraccaroli, Francesco Stefanni, Romeo Rizzi, Davide Quaglia, Franco Fummi:}
   \newblock Network Synthesis for Distributed Embedded Systems,
   \newblock {\it IEEE Trans. on Computers}
   \newblock 67(9) (2018) 1315--1330.
   % DOI:10.1109/TC.2018.2812797.

  \item {\sc Carlo Comin, Romeo Rizzi:}
   \newblock Checking dynamic consistency of conditional hyper temporal networks via mean payoff games: Hardness and (pseudo) singly-exponential time algorithm,
   \newblock {\it Inf. Comput.}
   \newblock 259(3) (2018) 348--374.

  \item {\sc Carlo Comin, Romeo Rizzi:}
    \newblock An Improved Upper Bound on Maximal Clique Listing via Rectangular Fast Matrix Multiplication
    \newblock {\it Algorithmica}
    \newblock 80(12) (2018) 3525--3562.
   
  \item {\sc Alessio Conte, Roberto Grossi, Andrea Marino, Romeo Rizzi:}
   \newblock Efficient enumeration of graph orientations with sources,
   \newblock {\it Discrete Applied Mathematics}
   \newblock 246 (2018) 22--37.

  \item {\sc Ademir Hujdurovic, Edin Husic, Martin Milanicw, Romeo Rizzi, Alexandru I. Tomescu:}
   \newblock Perfect Phylogenies via Branchings in Acyclic Digraphs and a Generalization of Dilworth's Theorem,
   \newblock {\it ACM Trans. Algorithms}
   \newblock 14(2) (2018) 20:1--20:26.

  \item {\sc Carlo Comin, Romeo Rizzi:}
   \newblock Improved Pseudo-polynomial Bound for the Value Problem and Optimal Strategy Synthesis in Mean Payoff Games,
   \newblock {\it Algorithmica}
   \newblock 77(4) (2017) 995--1021.

  \item {\sc Carlo Comin, Roberto Posenato, Romeo Rizzi:}
   \newblock Hyper temporal networks - A tractable generalization of simple temporal networks and its relation to mean payoff games,
   \newblock {\it Constraints}
   \newblock 22(2) (2017) 152--190.

  \item {\sc Franca Rinaldi, Romeo Rizzi:}
   \newblock Solving the train marshalling problem by inclusion-exclusion,
   \newblock {\it Discrete Applied Mathematics}
   \newblock 217 (2017) 685--690.

  \item {\sc Liliana Alcón, Marisa Gutierrez, István Kovács, Martin Milanic, Romeo Rizzi:}
   \newblock Strong cliques and equistability of EPT graphs,
   \newblock {\it Discrete Applied Mathematics}
   \newblock 203 (2016) 13--25.

  \item {\sc Both Emerite Neou, Romeo Rizzi, Stéphane Vialette:}
   \newblock Permutation Pattern matching in $(213, 231)$-avoiding permutations,
   \newblock {\it Discrete Mathematics \& Theoretical Computer Science}
   \newblock  18(2) (2016)
   
  \item {\sc David Cariolaro, Romeo Rizzi:}
   \newblock On the Complexity of Computing the Excessive $[B]$-Index of a Graph,
   \newblock {\it Journal of Graph Theory}
   \newblock 82(1) (2016) 65--74.

  \item {\sc Stefano Benati, Romeo Rizzi, Craig A. Tovey:}
   \newblock The complexity of power indexes with graph restricted coalitions,
   \newblock {\it Mathematical Social Sciences}
   \newblock 76 (2015) 53--63.

  \item {\sc Romeo Rizzi, Florian Sikora:}
   \newblock Some Results on More Flexible Versions of Graph Motif,
   \newblock {\it Theory Comput. Syst.}
   \newblock 56(4) (2015) 612--629.

  \item {\sc Alexandru I. Tomescu, Travis Gagie, Alexandru Popa, Romeo Rizzi, Anna Kuosmanen, Veli Mäkinen:}
   \newblock Explaining a Weighted DAG with Few Paths for Solving Genome-Guided Multi-Assembly,
   \newblock {\it IEEE/ACM Trans. Comput. Biology Bioinform.}
   \newblock 12(6) (2015) 1345--1354.

  \item {\sc Ferdinando Cicalese, Martin Milanic, Romeo Rizzi:}
   \newblock On the complexity of the vector connectivity problem,
   \newblock {\it Theor. Comput. Sci.}
   \newblock 591 (2015) 60--71.        

  \item {\sc Alberto Caprara, Mauro Dell'Amico, Jos\'e Carlos D\'\i{}az, Manuel Iori, Romeo Rizzi:}
   \newblock Friendly bin packing instances without Integer Round-up Property,
   \newblock {\it Math. Program.}
   \newblock  150(1) (2015) 5--17.

  \item {\sc Laurent Bulteau, Guillaume Fertin, Romeo Rizzi, St\'ephane Vialette:}
   \newblock  Some algorithmic results for [2]-sumset covers,
   \newblock {\it  Inf. Process. Lett.}
   \newblock  115(1) (2015) 1--5.

  \item {\sc Romeo Rizzi, David Cariolaro:}
   \newblock  Polynomial Time Complexity of Edge Colouring Graphs with Bounded Colour Classes,
   \newblock {\it Algorithmica}
   \newblock   69(3) (2014) 494--500.

  \item {\sc Romeo Rizzi, Alexandru I.~Tomescu, Veli M\"akinen:}
   \newblock  On the complexity of Minimum Path Cover with Subpath Constraints for multi-assembly,
   \newblock {\it  BMC Bioinformatics}
   \newblock  15(S--9) (2014) S5.

  \item {\sc Bostjan Bresar, Tanja Gologranc, Martin Milanic, Douglas F. Rall, Romeo Rizzi:}
   \newblock  Dominating sequences in graphs,
   \newblock {\it  Discrete Mathematics}
   \newblock  336 (2014) 22--36.

  \item {\sc Marien Abreu, Domenico Labbate, Romeo Rizzi, John Sheehan:}
   \newblock  Odd 2-factored snarks,
   \newblock {\it  Eur. J. Comb.}
   \newblock  36 (2014) 460--472.

  \item {\sc Guillaume Blin, Paola Bonizzoni, Riccardo Dondi, Romeo Rizzi, Florian Sikora:}
   \newblock  Complexity insights of the Minimum Duplication problem,
   \newblock {\it  Theor. Comput. Sci.}
   \newblock  530 (2014) 66--79.

  \item {\sc Martin Milanic, Romeo Rizzi, Alexandru I.~Tomescu:}
   \newblock  Set graphs. II. Complexity of set graph recognition and similar problems,
   \newblock {\it Theor. Comput. Sci.}
   \newblock   547 (2014) 70--81. 

  \item {\sc Alexandru I.~Tomescu, Anna Kuosmanen, Romeo Rizzi, Veli M\"akinen:}
   \newblock  A novel min-cost flow method for estimating transcript expression with RNA-Seq,
   \newblock {\it BMC Bioinformatics}
   \newblock   14(S-5) (2013) S15. 

  \item {\sc Guillaume Blin, Romeo Rizzi, Florian Sikora, St\'ephane Vialette:}
   \newblock  Minimum Mosaic Inference of a Set of Recombinants,
   \newblock {\it  Int. J. Found. Comput. Sci.}
   \newblock  24(1) (2013) 51--66.

  \item {\sc Romeo Rizzi, Alexandru I.~Tomescu:}
   \newblock  Ranking, unranking and random generation of extensional acyclic digraphs,
   \newblock {\it Inf. Process. Lett.}
   \newblock   113(5--6) (2013) 183--187. 

  \item {\sc Guillaume Blin, Romeo Rizzi, St\'ephane Vialette:}
   \newblock  A Faster Algorithm for Finding Minimum Tucker Submatrices,
   \newblock {\it  Theory Comput. Syst.}
   \newblock  51(3) (2012) 270--281.

  \item {\sc Romeo Rizzi, Luca Nardin:}
   \newblock  Polynomial Time Instances for the IKHO Problem,
   \newblock {\it ISRN Electronics}
   \newblock  2012, 10 pages (2012).

  \item {\sc Giulia Galbiati, Romeo Rizzi, Edoardo Amaldi:}
   \newblock  On the approximability of the minimum strictly fundamental cycle basis problem,
   \newblock {\it Discrete Applied Mathematics}
   \newblock  159(4) (2011) 187--200.

  \item {Marcin Kubica, Romeo Rizzi, St\'ephane Vialette, Tomasz Walen:}
   \newblock Approximation of RNA multiple structural alignment,
   \newblock {\it J. Discrete Algorithms}
   \newblock 9(4) (2011) 365--376.

  \item {\sc Paola Bonizzoni, Gianluca Della Vedova, Riccardo Dondi, Yuri Pirola, Romeo Rizzi:}
   \newblock  Pure Parsimony Xor Haplotyping,
   \newblock {\it IEEE/ACM Transactions on Computational Biology and Bioinformatics}
   \newblock  7(4) (2010) 598--609.

  \item {\sc David Cariolaro, Romeo Rizzi:}
   \newblock  Excessive factorizations of bipartite multigraphs,
   \newblock {\it Discrete Applied Mathematics}
   \newblock  158 (2010) 1760--1766.

  \item {\sc Ga\"elle Brevier, Romeo Rizzi, St\'ephane Vialette:}
   \newblock   Complexity issues in color-preserving graph embeddings,
   \newblock   {\it Theor. Comput. Sci.}
   \newblock    411(4-5) (2010) 716--729.

  \item {\sc Guillaume Fertin, Danny Hermelin, Romeo Rizzi, St\'ephane Vialette:}
   \newblock  Finding common structured patterns in linear graphs,
   \newblock {\it Theor. Comput. Sci.}
   \newblock 411(26--28) (2010) 2475--2486.

  \item {\sc Romeo Rizzi, Pritha Mahata, Luke Mathieson, Pablo Moscato:}
   \newblock  Hierarchical Clustering Using the Arithmetic-Harmonic Cut: Complexity and Experiments,
   \newblock {\it PLoS ONE}
   \newblock 5(12) (2010) .

  \item {\sc Romeo Rizzi:}
   \newblock   Minimum Weakly Fundamental Cycle Bases Are Hard To Find,
   \newblock {\it Algorithmica}
   \newblock 53(3) (2009) 402--424.

  \item {\sc Telikepalli Kavitha, Christian Liebchen, Kurt Mehlhorn, Dimitrios Michail, Romeo Rizzi, Torsten Ueckerdt, Katharina Anna Zweig:}
   \newblock  Cycle bases in graphs characterization, algorithms, complexity, and applications,
   \newblock   {\it Computer Science Review}
   \newblock   3(4) (2009) 199--243.

  \item {\sc Alan A.~Bertossi, Cristina M.~Pinotti, Romeo Rizzi:}
   \newblock  Optimal receiver scheduling algorithms for a multicast problem,
   \newblock {\it Discrete Applied Mathematics}
   \newblock  157(15) (2009) 3187--3197.

  \item {\sc Peter Biro, David Manlove, Romeo Rizzi:}
   \newblock   Maximum weight cycle packing in directed graphs, with application to kidney exchange programs,
   \newblock {\it Discrete Mathematics, Algorithms and Applications}
   \newblock  1(4) (2009) 499--517.

  \item {\sc Guillaume Fertin, Romeo Rizzi, St\'ephane Vialette:}
   \newblock  Finding Occurrences of Protein
              Complexes in Protein-Protein Interaction Graphs,
   \newblock {\it Journal of Discrete Algorithms}
   \newblock 7(1) (2009) 90--101.

  \item {\sc Ekkehard K\"ohler, Christian Liebchen, Gregor W\"unsch, Romeo Rizzi:}
   \newblock  Lower bounds for strictly fundamental cycle bases in grid graphs.    \newblock {\it Networks}
   \newblock 53(2) (2009) 191--205.

  \item {\sc Stefano Benati, Romeo Rizzi:}
   \newblock   The optimal statistical median of a convex set of arrays,
   \newblock {\it Journal of Global Optimization}
   \newblock 44(1) (2009) 79--97.

  \item {\sc Romeo Rizzi:}
   \newblock   Approximating the Maximum $3$-Edge-Colorable Subgraph Problem,
   \newblock {\it Discrete Mathematics}
   \newblock  309(12) (2009) 4164--4168.

  \item {\sc Richard C.~Brewster, Pavol Hell, Romeo Rizzi:}
   \newblock  Oriented star packings,
   \newblock {\it Journal of Combinatorial Theory, Series~B}
   \newblock  98 (2008) 558--576.

  \item {\sc Giuseppe Lancia, R.~Ravi, Romeo Rizzi:}
   \newblock  Haplotyping for Disease Association: A Combinatorial Approach, 
   \newblock {\it IEEE Transactions on Computational Biology and Bioinformatics}
   \newblock  5(2) (2008) 245--251.

  \item {\sc Danny Hermelin, Dror Rawitz, Romeo Rizzi, St\'ephane Vialette:}
   \newblock  The Minimum Substring Cover Problem,
   \newblock {\it Information and Computation}
   \newblock 206(11) (2008) 1303--1312.

  \item {\sc Reuven Cohen, Liran Katzir, Romeo Rizzi:}
   \newblock   On the Trade-off Between Energy and Multicast Efficiency in 802.16e-like Mobile Networks,
   \newblock {\it IEEE Transactions on Mobile Computing}
   \newblock  7(3) (2008) 346--357.
%   \newblock \\ - a previous version,
%                  with only Reuven Cohen and Romeo Rizzi as authors,
%                  was also accepted at Infocom~2006.

  \item {\sc Giuseppe Lancia, Franca Rinaldi, Romeo Rizzi:}
   \newblock  Flipping letters to minimize the support of a string,
   \newblock  {\it International Journal of Foundations of Computer Science}
   \newblock  19(1) (2008) 5--17.

  \item {\sc Guillaume Blin, Cedric Chauve, Guillaume Fertin, Romeo Rizzi, St\'ephane Vialette:}
   \newblock  Comparing Genomes with Duplications: A Computational Complexity Point of View.
   \newblock  {\it IEEE/ACM Trans. Comput. Biology Bioinform.}
   \newblock  4(4) (2007) 523--534.

  \item {\sc Michael Elkin, Christian Liebchen, Romeo Rizzi:}
   \newblock  New length bounds for cycle bases,
   \newblock {\it Information Processing Letters}
   \newblock  104(5) (2007) 186--193.

  \item {\sc Francesco Maffioli, Romeo Rizzi, Stefano Benati:}
   \newblock  Least and most colored bases,
   \newblock {\it Discrete Applied Mathematics}
   \newblock  155(15) (2007) 1958--1970.

  \item {\sc Stephen Finbow, Andrew King, Gary MacGillivray, Romeo Rizzi:}
   \newblock  The firefighter problem for graphs of maximum degree three,
   \newblock {\it Discrete Mathematics}
   \newblock  307(16) (2007) 2094--2105.

  \item {\sc Christian Liebchen, Romeo Rizzi:}
   \newblock  Classes of cycle bases,
   \newblock {\it Discrete Applied Mathematics}
   \newblock  155 (2007) 337--355.
  % gia inserito nel 2006

  \item {\sc Stefano Benati, Romeo Rizzi:}
   \newblock  A mixed integer linear programming formulation
              of the optimal mean/Value-at-Risk portfolio problem,
   \newblock {\it European Journal of Operational Research}
   \newblock  176 (2007) 423--434.
  % gia inserito nel 2006

  \item {\sc Alessandro Mei, Romeo Rizzi:}
   \newblock  Online Permutation Routing in
              Partitioned Optical Passive Star Networks,
   \newblock {\it IEEE Trans. Computers}
   \newblock  55(12) (2006) 1557--1571.

  \item {\sc Alessandro Mei, Romeo Rizzi:}
   \newblock  Hypercube Computations on Partitioned Optical
              Passive Stars Networks,
   \newblock {\it IEEE Trans. Parallel Distrib. Syst.}
   \newblock  17(6) (2006) 497--507.
%   \newblock \\ - a preliminary version appeared in: HiPC 2003, 95--104.

  \item {\sc Romeo Rizzi:}
   \newblock  Acyclically Pushable Bipartite Permutation Digraphs: an algorithm,
   \newblock {\it Discrete Mathematics}
   \newblock 306(12) (2006) 1177--1188.

  \item {\sc Romeo Rizzi, Marco Rospocher:}
   \newblock  Covering partially directed graphs with directed paths,
   \newblock {\it Discrete Mathematics}
   \newblock 306(13) (2006) 1390--1404.

  \item {\sc Giuseppe Lancia, Romeo Rizzi:}
   \newblock  A polynomial case of the parsimony haplotyping problem,
   \newblock {\it Oper. Res. Lett.}
   \newblock  34(3) (2006) 289--295.

  \item {\sc Guillaume Blin, Guillaume Fertin, Romeo Rizzi,
                  St\'ephane Vialette:}
   \newblock What Makes the Arc-Preserving Subsequence Problem Hard?
   \newblock  {\it Transactions on Computational Systems Biology II}
   \newblock  LNCS vol.~3680 (2005) 1--36.
%   \newblock \\ - a draft version of this work appeared on:
%                  International Conference on Computational Science (2) 2005: 860-868.

  \item {\sc Vineet Bafna, Sorin Istrail, Giuseppe Lancia, Romeo Rizzi:}
   \newblock  Polynomial and APX-hard cases of the Individual Haplotyping Problem,
   \newblock {\it Theoretical Computer Science}
   \newblock  335(1) (2005) 109--125.
%   \newblock \\ - a previous related work by the same authors
%               ``SNPs Problems: Complexity and Practical Algorithms''
%              was also accepted at WABI 2002.

  \item {\sc Zhi-Zhong Chen, Tao Jiang, Guohui Lin, Romeo Rizzi,
                  Jianjun Wen, Dong Xu, Ying Xu:}
   \newblock  More Reliable Protein NMR Peak Assignment via Improved $2$-Interval Scheduling,
   \newblock {\it Journal of Computational Biology}
   \newblock 12(2) 2005 129--146.

  \item {\sc Christian Liebchen, Romeo Rizzi:}
   \newblock  A greedy approach to compute a minimum cycle basis
              of a directed graph,
   \newblock {\it Information Processing Letters}
   \newblock  94(3) (2005) 107--112.
   %IPL3233
   %online via ScienceDirect:
   %http://authors.elsevier.com/TrackMyPaper.html?add_art=myarticles&trk_article=IPL3233&trk_mail=on&trk_surname=Liebchen

  \item {\sc Mauro Cettolo, Michele Vescovi, Romeo Rizzi:}
   \newblock  Evaluation of BIC-based algorithms for audio segmentation,
   \newblock {\it Computer Speech \& Language}
   \newblock 19(2) (2005) 147--170.
%   \newblock Volume 19, Issue 2, April (2005) pages 147-170
%   \newblock \\ - a previous work by the same authors
%                ``A DP Algorithm for Speaker Change Detection'',
%               a starting point for this subsequent work,
%               had been accepted at Eurospeech 2003.

  \item {\sc Elia~Ardizzoni, Alan A.~Bertossi, Maria Cristina Pinotti,
                  Shashank Ramaprasad,  Romeo Rizzi,  Madhusudana V.S. Shashanka:}
   \newblock  Optimal Skewed Data Allocation on Multiple Channels with Flat
              Broadcast per Channel,
   \newblock {\it IEEE Transactions on Computers}
   \newblock  54(5) (2005) 558--572.

  \item {\sc A.A. Bertossi, M.C. Pinotti, R. Rizzi, P. Gupta:} % Phalguni Gupta 
   \newblock  Allocating Servers in Infostations for Bounded Simultaneous Requests,
   \newblock {\it Journal of Parallel and Distributed Computing}
   \newblock  64 (2004) 1113--1126.
%   \newblock \\ - also accepted at IEEE Int'l Parallel and Distributed Processing Symposium, 2003.

  \item {\sc Alan A.~Bertossi, Cristina M.~Pinotti, Romeo Rizzi, Anil M.~Shende:}
   \newblock  Channel Assignment for Interference Avoidance in Honeycomb Wireless Networks,
   \newblock {\it Journal of Parallel and Distributed Computing}
   \newblock  64 (2004) 1329--1344.

  \item {\sc Alberto Caprara, Andrea Lodi, Romeo Rizzi:}
   \newblock  On $d$-Threshold Graphs and $d$-Dimensional Bin Packing,
   \newblock {\it Networks}
   \newblock  44(4) (2004) 266--280.
%   \newblock \\ - also accepted at 2003 Optimization Days.

  \item {\sc Alberto Caprara, Alessandro Panconesi, Romeo Rizzi:}
   \newblock  Packing Cuts in Graphs,
   \newblock {\it Networks}
   \newblock  44(1) (2004) 1--11.
%   \newblock \\ - part of this work was published in the proceedings of ESA 2001.

  \item {\sc Giuseppe Lancia, Maria Cristina Pinotti, Romeo Rizzi:}
   \newblock  Haplotyping Populations by Pure Parsimony: Complexity, 
              Exact, and Approximation Algorithms,
   \newblock {\it INFORMS J.~on Comp.}
   \newblock  16(4) (2004) 348--359.

  \item {\sc Michele Conforti, Romeo Rizzi:}  
   \newblock  Combinatorial Optimization
              - Polyhedra and efficiency: A book review,
   \newblock {\it 4OR}
   \newblock 2(2) (2004) 153--159.

  \item {\sc Alberto Caprara, Alessandro Panconesi, Romeo Rizzi:}
   \newblock  Packing Cycles in Undirected Graphs,
   \newblock {\it Journal of Algorithms}
   \newblock  48(1) (2003) 239--256.
%   \newblock \\ - part of this work was published in the proceedings of ESA 2001.
   % report DIT: NO - POLARIS: SI

  \item {\sc Romeo Rizzi:}
   \newblock  On Rajagopalan and Vazirani's $\frac{3}{2}$-Approximation
              Bound for the Iterated $1$-Steiner Heuristic,
   \newblock {\it Information Processing Letters}
   \newblock  86(6) (2003) 335--338.
   %IPL2862
   %online via ScienceDirect:
   %http://www.sciencedirect.com/science?_ob=GatewayURL&_origin=AUTHORALERT&_method=citationSearch&_piikey=S0020019003002102&_version=1&md5=5603f0c51caeab0a09ec923ed91eccbe
   % report DIT: NO - POLARIS: SI

  \item {\sc Alessandro Mei, Romeo Rizzi:}
   \newblock  Routing Permutations in Partitioned Optical Passive Stars Networks,
   \newblock {\it Journal of Parallel and Distributed Computing}
   \newblock  63(9) (2003) 847--852.
   \newblock \\ - also accepted at IPDPS 2002 where it received the {\bf Best Paper Award}.
   % report DIT: NO - POLARIS: SI

  \item {\sc Richard C.~Brewster, Romeo Rizzi:}
   \newblock  On the complexity of digraph packings,
   \newblock {\it Information Processing Letters}
   \newblock  86(2) (2003) 101--106.
   %IPL2829
   %online via ScienceDirect:
   %http://www.sciencedirect.com/science?_ob=GatewayURL&_origin=AUGATEWAY&_method=citationSearch&_piikey=S0020019002004787&_version=1&md5=4550693a5fda4e6a7d92fb51803d3114
   % report DIT: NO - POLARIS: SI

  \item {\sc Romeo Rizzi:}
   \newblock  A Simple Minimum $T$-Cut Algorithm,
   \newblock {\it Discrete Applied Mathematics}
   \newblock  129 (2003) 539--544.
   % report DIT: SI (senza dati rivista) - POLARIS: SI

  \item {\sc Richard C.~Brewster, Pavol Hell, Sarah H.~Pantel, Romeo Rizzi, Anders Yeo:}
   \newblock  Packing paths in digraphs,
   \newblock {\it Journal of Graph Theory}
   \newblock  44(2) (2003) 81--94.
   %Published Online: 2 Sep 2003 DOI: 10.1002/jgt.10126
   % report DIT: NO - POLARIS: SI

  \item {\sc Romeo Rizzi:}
   \newblock  Cycle cover property and $CPP=SCC$ property are not equivalent,
   \newblock {\it Discrete Mathematics}
   \newblock  259 (2002) 337--342.
   % report DIT: SI - POLARIS: SI

  \item {\sc Alberto Caprara, Romeo Rizzi:}
   \newblock  Packing Triangles in Bounded Degree Graphs,
   \newblock {\it Information Processing Letters}
   \newblock  84(4) (2002) 175--180.
   % report DIT: SI - POLARIS: SI

  \item {\sc Romeo Rizzi:}
   \newblock  Minimum $T$-cuts and optimal $T$-pairings,
   \newblock {\it Discrete Mathematics}
   \newblock  257(1) (2002) 177--181.
   % report DIT: SI - POLARIS: SI

  \item {\sc Romeo Rizzi:}
   \newblock  Finding $1$-factors in bipartite regular graphs,
              and edge-coloring bipartite graphs,
   \newblock {\it SIAM Journal on Discrete Mathematics}
   \newblock  15(3) (2002) 283--288. 
   % report DIT: SI - POLARIS: SI

  \item {\sc Alberto Caprara, Romeo Rizzi:}
   \newblock  Improved Approximation for Breakpoint Graph Decomposition
              and Sorting by Reversals,
   \newblock {\it Journal of Combinatorial Optimization}
   \newblock  6 (2002) 157--182.
   % report DIT: SI - POLARIS: SI

  \item {\sc Romeo Rizzi:}
   \newblock  Complexity of Context-free Grammars with Exceptions,
              and the inadequacy of grammars as models for XML and SGML,
   \newblock {\it Markup Languages: Theory and Practice}
   \newblock  3(1) (2001) 107--116.
   % report DIT: SI - POLARIS: SI

  \item {\sc Alessandro Panconesi, Romeo Rizzi:}
   \newblock  Some Simple Distributed Algorithms for Sparse Networks,
   \newblock {\it Distributed Computing}
   \newblock  14 (2001) 97--100.
   % report DIT: SI - POLARIS: SI

  \item {\sc Romeo Rizzi:}
   \newblock  On the Recognition of $P_4$-Indifferent Graphs,
   \newblock {\it Discrete Mathematics}
   \newblock  239 (2001) 161--169.
   % report DIT: SI - POLARIS: SI

  \item {\sc Romeo Rizzi:}
   \newblock  On $4$-connected graphs without even cycle decompositions,
   \newblock {\it Discrete Mathematics}
   \newblock  234 (2001) 181--186.
   % report DIT: SI - POLARIS: SI

  \item {\sc Romeo Rizzi:}
   \newblock  Excluding a simple good pair approach to directed cuts,
   \newblock {\it Graphs and Combinatorics}
   \newblock  17 (2001) 741--744.
   % report DIT: SI - POLARIS: SI

  \item {\sc Michele Conforti, Romeo Rizzi:}  
   \newblock  Shortest Paths in Conservative Graphs,
   \newblock {\it Discrete Mathematics}
   \newblock  226 (2001) 143--153.
%   \newblock \\ - part of this work
%                  was published in the proceedings of AIRO '96.
   % report DIT: SI - POLARIS: SI

  \item {\sc Romeo Rizzi:}
   \newblock  A note on range-restricted circuit covers,
   \newblock {\it Graphs and Combinatorics}
   \newblock  16 (2000) 355--358.
   % report DIT: SI - POLARIS: SI

  \item {\sc Romeo Rizzi:}
   \newblock  On minimizing symmetric set functions,
   \newblock {\it Combinatorica}
   \newblock  20(3) (2000) 445--450.
   % report DIT: SI - POLARIS: SI

  \item {\sc Romeo Rizzi:}
   \newblock  A short proof of K\H{o}nig's matching theorem,
   \newblock {\it Journal of Graph Theory}
   \newblock  33(3) (2000) 138--139.
   % report DIT: SI - POLARIS: SI

  \item {\sc Ajai Kapoor, Romeo Rizzi:}
   \newblock  Edge-coloring bipartite graphs,
   \newblock {\it Journal of Algorithms}
   \newblock  34(2) (2000) 390--396.
   % report DIT: SI - POLARIS: SI

  \item {\sc Romeo Rizzi:}
   \newblock  Indecomposable $r$-graphs and some other counterexamples,
   \newblock {\it Journal of Graph Theory}
   \newblock  32(1) (1999) 1--15.
   % report DIT: SI - POLARIS: SI

  \item {\sc Alberto Caprara, Romeo Rizzi:}  
   \newblock  Improving a Family of Approximation
              Algorithms to Edge Color Multigraphs,
   \newblock {\it Information Processing Letters}
   \newblock  68(1) (1998) 11--15.
   % report DIT: SI - POLARIS: SI

  \item {\sc Romeo Rizzi:}
   \newblock  K\H{o}nig's Edge Coloring Theorem without augmenting paths,
   \newblock {\it Journal of Graph Theory}
   \newblock  29 (1998) 87.

\end{etaremune}


%\newpage
\vspace{1.8mm}
\voice{{\LARGE Conferenze Internazionali con Referee}}

../commons/ListaConferenzeInternazionali.tex


La lista sopra ovviamente non vuole essere completa.
In particolare, ulteriori miei lavori sono stati pubblicati in proceedings
di conferenza nazionali e/o senza referaggio
o sono stati presentati in conferenze senza proceedings o raduni informali
pi\`u o meno prestigiosi.


\vspace{1.8mm}

\voice{{\LARGE Parti di Libro}}

../commons/ListaCapitoliLibro.tex


\vspace{1.8mm}

\voice{{\LARGE Riviste Nazionali}}

\begin{etaremune}
\vspace{-3.0mm}

  \item {\sc Romeo Rizzi},
   \newblock  Impaccando $T$-tagli e $T$-giunti,
   \newblock {\it Bollettino Sezione B dell'Unione Matematica Italiana},
   \newblock {Fascicolo speciale dedicato alle tesi di dottorato}
             (8) 1-A Suppl. (1998) 201-204.
   % report DIT: SI - POLARIS: SI

\end{etaremune}



\vspace{1.8mm}

%\newpage
\voice{{\LARGE Libri (per la didattica)}}

../commons/ListaLibri.tex



%\vspace{1.8mm}
%
%\voice{{\LARGE Articoli Accettati o Sottomessi}}
%
%\begin{itemize}
%\vspace{-3.0mm}


% Stephen Finbow, Andrew King, Gary MacGillivray, Romeo Rizzi,
% The Firefighter Problem for Graphs of Maximum Degree Three
% EuroComb'03

%  \vspace{1.2mm}
%  \item[5.] {\sc Romeo Rizzi},
%   \newblock  From SGML to XML: DTDs which do necessarily explode,
%   \newblock {\it submitted}. % to Markup Languages: Theory and Practice}

%  \vspace{1.2mm}
%  \item[6.] {\sc A.A. Bertossi, M.C. Pinotti, R. Rizzi},
%  \newblock  Channel Assignment with Separation on Trees and Interval Graphs,
%  \newblock {\it submitted}.
%  \newblock \\ - also accepted at 3rd Int'l Workshop on Wireless, Mobile and Ad Hoc Networks, 2003.

%  \vspace{1.2mm}
%  \item[7.] {\sc Sheila Ferneyhough, Gary MacGillivray, Romeo Rizzi},
%   \newblock  Constructing Dice with Given Dominance Digraph,
%   \newblock {\it submitted}. % to Information Processing Letters}

%  \vspace{1.2mm}
%  \item[8.] {\sc Marco Rospocher, Romeo Rizzi},
%   \newblock  All pairs shortest path: proposal and evaluation of two
%              simple and quick $O(n^3)$ algorithms,
%   \newblock {\it submitted}. % to Algorithmica}


%  \vspace{1.2mm}
%  \item[14.] {\sc Romeo Rizzi, Andr\'{a}s Seb\H{o}},
%   \newblock A note on Matroid Flows,
%   \newblock {\it in preparation}

%  \vspace{1.2mm}
%  \item[15.] {\sc Romeo Rizzi},
%   \newblock  On packing $T$-joins,
%   \newblock {\it in preparation}

%  \vspace{1.2mm}
%  \item[16.] {\sc Romeo Rizzi},
%   \newblock  Efficiently solvable extensions of the MS-matching problem,
%   \newblock {\it in preparation}

%  \vspace{1.2mm}
%  \item[17.] {\sc Romeo Rizzi},
%   \newblock  A new method for finding cuts of minimum weight,
%   \newblock {\it in preparation}

%  \vspace{1.2mm}
%  \item[18.] {\sc Romeo Rizzi},
%   \newblock  The $T$-join cone,
%   \newblock {\it in preparation}

%  \vspace{1.2mm}
%  \item[19.] {\sc Romeo Rizzi},
%   \newblock  Efficient implementations of greedy order algorithms,
%   \newblock {\it under revision} %  to Mathematical Programming}


%\end{itemize}


%\vspace{1.8mm}
%
%\voice{{\LARGE Rapporti Interni} (contributi non apparsi altrove)}
%
%\begin{itemize}
%\vspace{-3.0mm}
%  \item[1.] {\sc Romeo Rizzi},
%   \newblock  Simple proofs for the bipartite matching politope,
%   \newblock {\it Rapporto Interno n.5 - 16 maggio 1997},
%   \newblock {Dipartimento di Matematica Pura ed Applicata},
%   \newblock {Universit\`a di Padova}, Italy.
%   \newblock \\ Composto ad uso proprio, mai sottomesso.
%
%  \vspace{1.2mm}
%  \item[2.] {\sc Romeo Rizzi},
%   \newblock  A proof to Schwarz's conjecture
%              on strongly connected graphs,
%   \newblock {\it UTM 607 - November 2001},
%   \newblock {Dipartimento di Matematica},
%   \newblock {Universit\`a di Trento}, Italy.
%   \newblock \\ Dopo averlo composto (e presentato ad un seminario
%                in Bologna),
%                venni a sapere, tramite uno scambio
%                epistolare  con il prof.~Schwartz,
%                che il risultato era gi\`a stato
%                da poco ottenuto dal  prof.~Ron Aharoni del Technion.
%                Poich\`e la dimostrazione a richiesta
%                gentilmente inviatami da Aharoni era in tutto
%                superiore alla mia, scelsi di non sottometterlo.
%
%\end{itemize}



\vspace{1.8mm}

\voice{{\LARGE Incarichi ed Onori}}

   - Dal 2004, opero come Area Editor per la rivista scientifica 4OR.

   - Editor dei proceedings del meeting
   in Graph Theory tenuto ad Oberwolfach nel gennaio 2003,
   ed organizzato da Reinhard Diestel, Alexander Schrijver
   e Paul D. Seymour.

   - Stesura, sotto la guida del Prof.~Michele Conforti,
   della presentazione su {\em 4OR} dell'opera
   {\em ``Combinatorial Optimization
              - Polyhedra and efficiency''}
   di Alexander Schrijver.

   - Best Paper Award ad IPDPS 2002
     per un lavoro in collaborazione con Alessandro Mei.

   - Ho tenuto il corso
    ``Algorithmic and Complexity issues in Structure Prediciton and/or Determination''
     alla Third International School on Biology,
     Computation and Information (BCI 2006).
     Dobbiaco (BZ), Italy, September 11-15, 2006.
    %home page della scuola: http://bioinf.dimi.uniud.it/bci2006/

   - Organizzatore di una invited session in Computational Biology ad AIRO 2005.
     %  (coorganizzatore: Giuseppe Lancia).

   - Invited speaker (International Keynote Speaker) a BioInfoSummer 2004.
Australian National University. December 6-10, 2004. Canberra.

   - Invited speaker al
     ``Workshop on Cycle and Cut Bases'' (14-16 maggio 2008)
     tenutosi a T\"ubingen ed inserito nel quadro
     SPP 1126 (Algorithmik großer und komplexer Netzwerke).
%     http://www-pr.informatik.uni-tuebingen.de/workshops/WorkshopOnCycleBasesMay2007/

   - Invited speaker al
   ``28 th Ljubljana – Leoben Graph Theory Seminar'',
   3-5 September 2014. Kooper (Capodistria). 

   - Reviewer per i Mathematical Reviews dell'American Mathematical Society
     dal 2004.

   - La mia biografia \`e inclusa nelle edizioni 2007 e 2009
     di \emph{Who's Who in the World}.

   - La mia biografia \`e inclusa nell'edizione 2009/2010
     di \emph{Outstanding Intellectuals of the 21st Century} (IBC, Cambridge).

   - Erd\"os number: 2.

   - Come educatore, ho ricevuto i seguenti riconoscimenti
     da parte dell'\emph{International Biographical Center, Cambridge}:

   \ \ \ - sono stato nominato \emph{International Educator of the Year} per il 2007 e 2009.

   \ \ \ - sono stato inserito nelle liste \emph{Top 100 Educators} 2008
           e 2009.

   \ \ \ - ho ricevuto \emph{The Decree of Excellence in Education}.

   
      % ma forse anche in:
      % Who's who in Europe
      % Who's who in Italy
      % Who's Who in Science in the World.
      % Who's Who in Science and Engineering.
      % Bertossi di se scriveva: His biography is included in the 1999 edition of Who's Who in the World


%\vspace{1.8mm}
%
%\voice{{\LARGE Progetti software}}
%
%\begin{itemize}
%\vspace{-3.0mm}
%  \item[1.] {\sc Romeo Rizzi},
%   \newblock  {\em A simple snark recognizer}.
%   \newblock  Si tratta di 
%             un progetto ``branch and cut''
%             sotto ABACUS, per il riconoscimento
%             di $3$-grafi i cui archi non possono essere colorati
%             in tre colori.
%\end{itemize}




\vspace{1.2cm}
Verona \hspace{7.8cm} Romeo Rizzi
PDPS 2002
     per un lavoro in collaborazione con Alessandro Mei.

   - Ho tenuto il corso
    ``Algorithmic and Complexity issues in Structure Prediciton and/or Determination''
     alla Third International School on Biology,
     Computation and Information (BCI 2006).
     Dobbiaco (BZ), Italy, September 11-15, 2006.
    %home page della scuola: http://bioinf.dimi.uniud.it/bci2006/

   - Organizzatore di una invited session in Computational Biology ad AIRO 2005.
     %  (coorganizzatore: Giuseppe Lancia).

   - Invited speaker (International Keynote Speaker) a BioInfoSummer 2004.
Australian National University. December 6-10, 2004. Canberra.

   - Invited speaker al
     ``Workshop on Cycle and Cut Bases'' (14-16 maggio 2008)
     tenutosi a T\"ubingen ed inserito nel quadro
     SPP 1126 (Algorithmik großer und komplexer Netzwerke).
%     http://www-pr.informatik.uni-tuebingen.de/workshops/WorkshopOnCycleBasesMay2007/

   - Invited speaker al
   ``28 th Ljubljana – Leoben Graph Theory Seminar'',
   3-5 September 2014. Kooper (Capodistria). 

   - Reviewer per i Mathematical Reviews dell'American Mathematical Society
     dal 2004.

   - La mia biografia \`e inclusa nelle edizioni 2007 e 2009
     di \emph{Who's Who in the World}.

   - La mia biografia \`e inclusa nell'edizione 2009/2010
     di \emph{Outstanding Intellectuals of the 21st Century} (IBC, Cambridge).

   - Erd\"os number: 2.

   - Come educatore, ho ricevuto i seguenti riconoscimenti
     da parte dell'\emph{International Biographical Center, Cambridge}:

   \ \ \ - sono stato nominato \emph{International Educator of the Year} per il 2007 e 2009.

   \ \ \ - sono stato inserito nelle liste \emph{Top 100 Educators} 2008
           e 2009.

   \ \ \ - ho ricevuto \emph{The Decree of Excellence in Education}.

   
      % ma forse anche in:
      % Who's who in Europe
      % Who's who in Italy
      % Who's Who in Science in the World.
      % Who's Who in Science and Engineering.
      % Bertossi di se scriveva: His biography is included in the 1999 edition of Who's Who in the World


%\vspace{1.8mm}
%
%\voice{{\LARGE Progetti software}}
%
%\begin{itemize}
%\vspace{-3.0mm}
%  \item[1.] {\sc Romeo Rizzi},
%   \newblock  {\em A simple snark recognizer}.
%   \newblock  Si tratta di 
%             un progetto ``branch and cut''
%             sotto ABACUS, per il riconoscimento
%             di $3$-grafi i cui archi non possono essere colorati
%             in tre colori.
%\end{itemize}




\vspace{1.2cm}
Verona \hspace{7.8cm} Romeo Rizzi
