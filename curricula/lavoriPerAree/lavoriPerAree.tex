\documentclass[10pt]{article}

\usepackage{latexsym}
\usepackage[italian]{babel}

\textwidth 15.5cm
\textheight 23.2cm
\topmargin -1cm
\evensidemargin 0in
\oddsidemargin 0in


\def\G{{\cal G}}
\def\coNP{coN\!P}
\def\NP{N\!P}


\begin{document}

\title{\vspace*{-2.8cm}
            {\em I miei lavori}\\
            {\large descrizione e collocazione per aree di interesse}}
%           123456789012345678901234567890123456789012345678901234567890
%                    1         2         3         4         5         6    

\author{Romeo Rizzi}
\date{{\bf BRICS}\thanks{%
  \protect\begin{tabular}[t]{@{}l@{}}
     Basic Research in Computer Science,\protect\\
     Centre of the Danish National Research Foundation.
  \protect\end{tabular}}\\
  Department of Computer Science\\
  University of Aarhus\\
  Ny Munkegade\\
  DK-8000 Aarhus C, Denmark\\
\vspace*{2pt}
  {e-mail: \ \tt romeo$@$cwi.nl}}

\maketitle

\vspace*{-20pt}



\section{Introduzione: Interessi ed Aree di Ricerca}

A ben guardare,
tutti i miei lavori possono essere considerati
come lavori in Ricerca Operativa.
Pi\'u specificatamente l'interesse si
accentra sull'Ottimizzazione Combinatoria
e l'approccio \'e quasi sempre algoritmico.
Siccome poi la maggior parte dei
miei lavori ha qualche legame diretto
od indiretto con i grafi,
ho deciso di organizzare qui
i miei lavori solamente secondo le seguenti
aree di interesse:\\

Algoritmi Approssimati.
Algoritmi Distribuiti.
Algoritmi Euristici.
Complessit\'a Computazionale.
Matroidi.
Teoria e problemi di matching.
Circuit Double Cover Conjecture.
Grafi Perfetti.
Packing e Covering.
Colorazione di archi.
Fattorizzazione di grafi.
Problemi di cammini minimi.
Problemi di taglio minimo.
Biologia Computazionale.\\



Tengo tuttavia a precisare che queste non sono
le sole aree che stimolino il mio
interesse e la mia curiosit\'a scientifica.

La descrizione di uno stesso lavoro
sar\'a ripetuta ove esso venga incluso in pi\'u aree.
Ci\'o espander\'a considerevolmente la lista
ma ne faciliter\'a l'accesso a chi abbia un interesse specifico.\\








\section{Algoritmi Approssimati}

\begin{itemize}
  \vspace{1.4mm}
  \item[] {\sc Alberto Caprara, Romeo Rizzi},
   \newblock  Improved Breakpoint Graph Decompositions,
   \newblock {\em submitted}\\
{\bf descrizione:}
Il sorting by reversals \'e uno di quei
problemi in biologia computazionale
che sono venuti alla
luce con lo sviluppo del Progetto Genoma Umano.
L'interesse per questo problema da parte
della biologia computazionale \'e forte.
Caprara aveva purtroppo dimostrato che il problema
era NP-hard.
Sorge a questo punto l'interesse per degli algoritmi
di tipo approssimato.
Questi sono intesi a produrre delle soluzioni che, se non ottime,
siano quantomeno di provata qualit\'a,
ossia di qualit\'a superiore ad un certo bound fissato.
Noi si \'e migliorato il miglior
bound attuale.
Il tutto a prezzo di un'analisi combinatorica
di non poco peso per il lettore e per chi mai
volesse implementare l'algoritmo.
La complessit\'a computazionale dell'algoritmo
tuttavia non ne ha sofferto.\\

  \vspace{1.4mm}
  \item[] {\sc Romeo Rizzi},
   \newblock  On the Steiner tree $\frac{3}{2}$-approximation
              for quasi-bipartite graphs,
   \newblock {\em submitted} \\
{\bf descrizione:}
Dato un insieme di nodi,
un albero di Steiner deve toccarli tutti.
Il problema \'e quello di trovare un albero
di Steiner di peso minimo.
Questo problema,
originariamente studiato da Gauss,
\'e NP-hard.
La ricerca di algoritmi approssimati
per questo problema \'e fervida ed appassionata.
In un lavoro assai di voga,
Rajagopalan e Vazirani avevano
proposto un algoritmo $(\frac{3}{2}+\epsilon)$-approssimato
nel caso di grafi quasi bipartiti,
ossia quando ogni arco ha almeno un'estremit\'a
in un nodo richiesto.
Noi diamo un algoritmo $\frac{3}{2}$-approssimato
ed assai pi\'u semplice per lo stesso problema.
Inoltre, contrariamente all'algoritmo di Rajagopalan e Vazirani,
la nostra soluzione \'e effettivamente pratica.\\

  \vspace{1.4mm}
  \item[] {\sc Alberto Caprara, Romeo Rizzi},  
   \newblock  Improving a Family of Approximation
              Algorithms to Edge Color Multigraphs,
   \newblock {\em Inf. Proc. Lett.} 68 (1998) 11-15.\\
{\bf descrizione:}
Coloriamo gli archi di un grafo
in modo che archi incidenti ad uno stesso nodo
abbiano colori diversi.
Evidentemente non possiamo sperare di impiegare
meno di $\Delta$ colori,
ove $\Delta$ \'e il massimo grado di un nodo.
Il teorema di Vizing dice che
possiamo sempre farcela
con $\Delta+1$ colori.
Tuttavia ci\'o non \'e pi\'u vero
ove si consideri la possibilit\'a di avere archi paralleli
tra una stessa copia di nodi.
\'E stato tuttavia definito
un parametro $\Delta'$,
di facile computazione,
che costituisce una miglior stima inferiore
che non $\Delta$ sul numero di colori necessari.
Delle famigerate congetture
degli anni 80
affermano poi che ce la si pu\'o sempre
fare con al pi\'u $\Delta'+1$ colori.
In ogni caso,
il problema di colorare
gli archi con pochi colori ha importanti applicazioni.
Questo spiega la serie di algoritmi che sono stati
via via presentati per impiegare al pi\'u
$\alpha \Delta'$ colori.
Purtroppo, ogni nuovo algoritmo se ne esce
con un valore di $\alpha$ pi\'u vicino ad 1,
ma anche con una complessit\'a computazionale
e di implementazione sempre maggiore.
Noi abbiamo mostrato come con un semplice trucco fosse possibile
migliorare il grado di approssimazione per ciascuno
degli algoritmi sinora proposti senza incrementarne
la complessit\'a.
Ci\'o consente di dedurre la validit\'a della congettura
di cui sopra per $\Delta\leq 12$.\\

  \vspace{1.4mm}
  \item[] {\sc Alberto Caprara, Alessandro Panconesi, Romeo Rizzi},
   \newblock  Packing triangles in planar graphs,
   \newblock {\em ready for submission}\\
{\bf descrizione:}
Dato un grafo planare,
vogliamo trovare in esso il massimo numero di triangoli disgiunti.
A seconda che li si voglia solamente disgiunti sui nodi od
anche sugli archi abbiamo due distinti problemi.
Tuttavia, in una trattazione assai unificata,
diamo dei PTAS per entrambi i problemi
avvalendoci del teorema separatore.
Un PTAS \'e un algoritmo approssimante
a qual si voglia precisione
ed \'e in pratica il meglio che si possa sperare
quando il problema in questione
risulti essere NP-hard.
In effetti noi si dimostra inoltre che entrambi i problemi sono
NP-hard.
Precedentemente solamente il caso di triangoli disgiunti
sui nodi era noto essere tale e la dimostrazione
basava su una ben pi\'u complessa riduzione
che non quella da noi proposta.\\

  \vspace{1.4mm}
  \item[] {\sc Alberto Caprara, Alessandro Panconesi, Romeo Rizzi},
   \newblock  Packing Cuts in Graphs,
   \newblock {\em in preparation}\\
{\bf descrizione:}
Diamo svariati risultati di inapprossimabilit\'a
per il problema di trovare il maggior numero possibile di
tagli disgiunti in un dato grafo.
La questione dell' NP-hardness del problema
era stata recentemente posta al centro dell'attenzione
da alcuni lavori di Ajeev,
ma segue ora come conseguenza di questi ben pi\'u forti risultati.
(Non approssimabilit\'a entro un fattore $\log n$ nel caso generale
ed APX-hardness in svariati casi particolari).
Diamo altres\'\i\ degli algoritmi approssimati che mostrano
come tali risultati siano stretti.\\
\end{itemize}



\section{Algoritmi Distribuiti}

\begin{itemize}
  \vspace{1.4mm}
  \item[] {\sc Alessandro Panconesi, Romeo Rizzi, Riccardo Silvestri},
   \newblock  Distributed Computing in Sparse Networks,
   \newblock {\em ready for submission}\\
{\bf descrizione:}
Si consideri un sistema distribuito
ossia una rete dove i nodi rappresentano processori
e gli archi rappresentano collegamenti tra processori.
La rete \'e chiamata a computare
una funzione della propria stessa topologia,
come ad esempio:
un insieme
massimale indipendente,
una colorazione degli archi in al pi\'u $2\Delta -1$ colori,
una colorazione dei nodi in al pi\'u $\Delta +1$ colori.
Vogliamo che ci\'o avvenga in un numero di passi
polilogaritmico nel numero di nodi della rete
e tramite un protocollo deterministico.
Questa \'e una problematica che ha lungamente resistito
agli attacchi di esperti ed affermati ricercatori.
Noi si \'e aggirata la difficolt\'a considerando
il caso di reti sparse.
Infatti da un lato questo \'e il caso che
si presenta nelle applicazioni
e dall'altro qualora la rete fosse sufficientemente densa
i problemi di cui sopra ammetterebbero una soluzione
banale in pratica non distribuita.
Sotto questa ipotesi semplificatrice abbiamo mostrato
come si possano ottenere dei risultati positivi
per i problemi di cui sopra.\\

\end{itemize}



\section{Algoritmi Euristici}

\begin{itemize}
  \vspace{1.4mm}
  \item[] {\sc Roberto Battiti, Alan A.~Bertossi, Romeo Rizzi},  
   \newblock  Randomized Greedy Algorithms 
              for the Hypergraph Partitioning Problem,
   \newblock Proceedings of the DIMACS Workshop on
             Randomization methods in
             algorithm design (Princeton, NJ, 1997),
             edited by: Panos Pardalos, Sanguthevar Rajasekaran, 
                        and Jose Rolim, 
             American Mathematical Society,  21--35, 1998.
   \newblock DIMACS Ser. Discrete Math. Theoret. Comput. Sci., 43,
             Amer. Math. Soc., Providence, RI, 1999.\\
{\bf descrizione:}
Quando si mette un circuito su piastra
si vorrebbero piazzare met\'a dei componenti su
una faccia e met\'a sull'altra,
al tempo stesso minimizzando per\'o il numero
di collegamenti che debbano poi attraversare la piastra
tramite foro o ponte.
Pi\'u in generale, un problema tipico in VLSI
\'e quello di piazzare met\'a oggetti da una parte
e met\'a dall'altra (=partizionare),
cercando di rompere il minor numero possibile di legami.
Quando ogni legame riguarda coppie di nodi singoli
questo problema \'e stato catalogato come
GRAPH PARTITIONING ed \'e stato uno dei primi
ad essere dimostrato NP-completo.
Ma recentemente alcuni problemi nelle applicazioni
sembravano abbisognare di una pi\'u fedele rappresentazione
come problemi di HYPERGRAPH PARTITIONING,
dove sono degli assiemi di oggetti di cardinalit\'a
possibilmente maggiore di due
a non voler essere separati.
Ovviamente questo complica ulteriormente il problema.
In questo lavoro abbiamo proposto delle semplici euristche
di tipo greedy ed abbiamo osservato
come esse fossero competitive nei confronti di
ben pi\'u sofisticate euristiche proposte
da altri ricercatori.\\
\end{itemize}



\section{Complessit\'a Computazionale}

\begin{itemize}
\vspace{1.4mm}
  \item[] {\bf tesi di laurea,}
          relatore: {\sc Prof.~Francesco Maffioli}
          (Dipartimento di Elettronica, Politecnico di Milano).
          Innanzittutto devo segnalare che la
          mia tesi di laurea conteneva svariati risultati di NP-completezza.
          Per vicissitudini varie,
          dopo la tesi non mi sono occupato di questioni
          di complessit\'a per un lungo periodo
          e credo del materiale sia rimasto sepolto nella tesi.\\

  \vspace{1.4mm}
  \item[] {\sc Alberto Caprara, Alessandro Panconesi, Romeo Rizzi},
   \newblock  Packing Cuts in Graphs,
   \newblock {\em in preparation}\\
{\bf descrizione:}
Diamo svariati risultati di inapprossimabilit\'a
per il problema di trovare il maggior numero possibile di
tagli disgiunti in un dato grafo.
La questione dell' NP-hardness del problema
era stata recentemente posta al centro dell'attenzione
da alcuni lavori di Ajeev,
ma segue ora come conseguenza di questi ben pi\'u forti risultati.
(Non approssimabilit\'a entro un fattore $\log n$ nel caso generale
ed APX-hardness in svariati casi particolari).
Diamo altres\'\i\ degli algoritmi approssimati che mostrano
come tali risultati siano stretti.\\

  \vspace{1.4mm}
  \item[] {\sc Alberto Caprara, Alessandro Panconesi, Romeo Rizzi},
   \newblock  Packing triangles in planar graphs,
   \newblock {\em ready for submission}\\
{\bf descrizione:}
Dato un grafo planare,
vogliamo trovare in esso il massimo numero di triangoli disgiunti.
A seconda che li si voglia solamente disgiunti sui nodi od
anche sugli archi abbiamo due distinti problemi.
Tuttavia, in una trattazione assai unificata,
diamo dei PTAS per entrambi i problemi
avvalendoci del teorema separatore.
Un PTAS \'e un algoritmo approssimante
a qual si voglia precisione
ed \'e in pratica il meglio che si possa sperare
quando il problema in questione
risulti essere NP-hard.
In effetti noi si dimostra inoltre che entrambi i problemi sono
NP-hard.
Precedentemente solamente il caso di triangoli disgiunti
sui nodi era noto essere tale e la dimostrazione
basava su una ben pi\'u complessa riduzione
che non quella da noi proposta.\\
\end{itemize}



\section{Matroidi}

\begin{itemize}
  \vspace{1.4mm}
  \item[] {\sc Romeo Rizzi, Andr\'{a}s Seb\H{o}},
   \newblock A note on Matroid Flows,
   \newblock {\em in preparation}\\
{\bf descrizione:}
Il lavoro comincia con l'osservare
come una per altro importante
congettura di Seymour riguardante
una certa classe di matroidi
fosse di fatto mal posta.
Infatti,
ove la congettura fosse presa un po' troppo alla lettera,
diamo un controesempio alla stessa.
Forniamo quindi la versione rivista della congettura
e ne esploriamo le relazioni con altre
congetture e risultati.\\
\end{itemize}



\section{Teoria e problemi di matching}

\begin{itemize}
  \vspace{1.4mm}
  \item[] {\sc Romeo Rizzi},
   \newblock  A short proof of K\H{o}nig's matching theorem,
   \newblock {\em accepted by Journal of Graph Theory}\\
{\bf descrizione:}
``In ogni grafo bipartito la massima
cardinalit\'a di un matching \'e pari
alla minima cardinalit\'a di un node cover.''
-- qui lo si dimostra succintamente senza ricorrere
ai cammini aumentanti.
La stessa dimostrazione consente
inoltre di ottenere in tutta semplicit\'a
la ben pi\'u impegnativa generalizzazione di Egerv\'ary
al caso pesato.\\ 

  \vspace{1.4mm}
  \item[] {\sc Romeo Rizzi},
   \newblock  Indecomposable $r$-graphs and some other counterexamples,
   \newblock {\it Journal of Graph Theory} 32 (1) (1999) 1--15.\\
{\bf descrizione:}
Un $r$-grafo
\'e un punto intero nel cono
generato proiettando il politopo dei matching
perfetti in un grafo completo.
Noi diamo dei controesempi a delle congetture
di Seymour e di altri ricercatori
e mostriamo come le altezze per le basi
di Hilbert di questi coni non siano limitate
da costante alcuna.
Paul Seymour in persona mi ha chiesto
di sottomettere questo lavoro al suo giornale
spendendo significativi apprezzamenti.\\

  \vspace{1.4mm}
  \item[] {\sc Richard C.~Brewster, Pavol Hell, Sarah H.~Pantel, Romeo Rizzi, Anders Yeo},
   \newblock  Packing paths in digraphs,
   \newblock {\em accepted by Journal of Graph Theory}\\
{\bf descrizione:}
Un matching pu\'o essere considerato
come una collezione di copie di $K_2$
disgiunte sui nodi.
Una generalizzazione del probema di trovare un matching
di massima cardinalit\'a \'e pertanto quella di,
fissata una famiglia di grafi $\G$,
e dato un certo grafo in input $G$,
trovare una collezione di sottografi di $G$ disgiunti sui nodi,
tutti isomorfi a qualche membro di $\G$.
Questo problema, noto come il $\G$-packing problem,
ha ricevuto molta attenzione nella letteratura.
Tuttavia poco era noto nel caso $\G$
sia una famiglia di grafi diretti,
se non che questa
cornice fosse pi\'u generale.
In questo lavoro si \'e dato inizio all'esplorazione
del caso in cui $\G$
\'e una famiglia di grafi diretti.
In particolare, si \'e affrontato
e completamente caratterizzato
il caso in cui $\G$ \'e una famiglia di cammini diretti.
Il problema di coprire il maggior numero di nodi di
$G$ \'e risolvibile in tempo
polinomiale se $\G$ \'e riducibile
o alla famiglia $\G=\{\vec{P}_1\}$
o alla famiglia $\G=\{\vec{P}_1,\vec{P}_2\}$.
In tutti gli altri casi il
problema risulta essere $\NP$-completo.\\

  \vspace{1.4mm}
  \item[] {\sc Romeo Rizzi},
   \newblock  The $T$-join cone,
   \newblock {\em in preparation}\\
{\bf descrizione:}
Forniamo la descrizione del cono
dei $T$-joins tramite un sistema di diseguaglianze
lineari. Ci\'o generalizza la ben nota
descrizione di Seymour per il cono dei cicli.\\

  \vspace{1.4mm}
  \item[] {\sc Romeo Rizzi},
   \newblock  Efficiently solvable extensions of the MS-matching problem,
   \newblock {\em in preparation}\\
{\bf descrizione:}
Vogliamo accoppiare macchine e lavori con
il vicolo che se un certo lavoro (slave)
viene eseguito allora anche un altro lavoro (il suo master)
viene eseguito.
Obbiettivo: trovare un accoppiamento che massimizzi il numero
di lavori assegnati a macchine.
Questo \'e un problema squisitamente di Ricerca Operativa.
Purtroppo il problema \'e NP-hard in generale
e tuttavia \'e di interesse risolverlo anche in casi
particolari.
Una formula min-max era nota per un caso particolare.
Noi estendiamo l'ambito di validit\'a di tale formula
e consideriamo generalizzazioni di casi risolvibili
mostrando come esse stesse siano risolvibili.\\

  \vspace{1.4mm}
  \item[] {\sc Romeo Rizzi},
   \newblock  K\H{o}nig's Edge Coloring Theorem without augmenting paths,
   \newblock {\it Journal of Graph Theory} 29 (1998) 87.\\
{\bf descrizione:}
Il teorema dei balli \'e uno di
quei risultati con cui K\H{o}nig
ha posto le basi della teoria dei grafi per come oggi \'e conosciuta.
Il teorema ha applicazioni dirette in problemi di scheduling e timetabling,
ma anche in discipline pi\'u lontane come la statistica.
Nonostante la sua importanza, il risultato veniva derivato
come corollario di un'altro fondamentale teorema di K\H{o}nig.
Noi introduciamo  un'operazione elementare che
trasforma un grafo bipartito e $\Delta$-regolare
in un suo fratellino minore.
La dimostrazione \'e ora per induzione
ed \'e diretta.\\

  \vspace{1.4mm}
  \item[] {\sc Romeo Rizzi},
   \newblock  Finding $1$-factors in bipartite regular graphs,
              and edge-coloring bipartite graphs,
   \newblock {\em submitted} \\
{\bf descrizione:}
Questo lavoro migliora ulteriormente
lo stato del problema di colorare gli archi in un grafo
bipartito con al pi\'u $\Delta$ colori.
Il collo di bottiglia per la complessit\'a
computazionele dell'algoritmo proposto
da Kapoor
e Rizzi in ``Edge-coloring bipartite graphs''
era la procedura presa in prestito da Hopcroft e Cole
per il reperimento di un accoppiamento perfetto
in un grafo bipartito e regolare.
Qui si migliora su questa procedura
grazie a degli argomenti algebrici.
In lavoro mette in luce il legame tra
la fattorizzazione di grafi e la fattorizzazione
di numeri interi.\\

  \vspace{1.4mm}
  \item[] {\sc Romeo Rizzi},
   \newblock  Simple proofs for the bipartite matching politope,
   \newblock {\em Rapporto Interno n.5},
   \newblock {Dipartimento di Matematica Pura ed Applicata},
   \newblock {Universit\`a di Padova}, Italia.\\
{\bf descrizione:}
Questo rapporto interno vuole offrire
una trattazione semplice ed elegante
per il problema del matching su grafi bipartiti.
Abbiamo raccolto qui
alcune nostre dimostrazioni originali
ed altre prese in prestito.
Il lavoro va inteso come sussidio alla didattica.
I principali risultati del settore
vengono ottenuti senza sforzo.\\

  \vspace{1.4mm}
  \item[] {\sc Romeo Rizzi},
   \newblock  Minimum $T$-cuts and optimal $T$-pairings,
   \newblock {\em submitted}\\
{\bf descrizione:}
Diamo una formula di min-max
per la minima cardinalit\'a di un $T$-taglio.
Mostriamo inoltre come nel minimal TDI system
per il poliedro dei $T$-tagli
siano presenti disuguaglianze a coefficienti comunque grandi.\\

  \vspace{1.4mm}
  \item[] {\sc Romeo Rizzi},
   \newblock  A Simple Minimum $T$-Cut Algorithm,
   \newblock {\em in preparation}\\
{\bf descrizione:}
Diamo un' algoritmo diretto per il computo
di un minimo $T$-taglio.
L'algoritmo, pur basandosi sulle stesse propriet\'a
di submodularia\'a ed uncrossing al pari
delle soluzioni precedenti,
purtuttavia presenta dei vantaggi
di semplicit\'a e praticit\'a.
L'algoritmo reperisce inoltre
un $T$-accoppiamento ottimo,
fornendo quindi dimostrazione algoritmica
della ``buona caratterizzazione''
da noi fornita nel lavoro
``Minimum $T$-cuts and optimal $T$-pairings''.\\
\end{itemize}



\section{Circuit Double Cover Conjecture}

\begin{itemize}
  \vspace{1.4mm}
  \item[] {\sc Romeo Rizzi},
   \newblock  A note on range-restricted circuit covers,
   \newblock {\em accepted by Graphs and Combinatorics}\\
{\bf descrizione:}
Una delle pi\'u belle ed ambite congetture
in teoria dei grafi afferma che per ogni grafo
connesso che non possa essere separato
con la rimozione di un solo arco esiste
una famiglia di circuiti tali
che ogni arco del grafo
\'e contenuto in precisamente due di questi.
%La congettura implicherebbe ad esempio il ben noto
%teorema dei 4 colori.
Quando una questione \'e difficile
i matematici amano scavarle un fosso tutt'attorno
andando ad investigare questioni affini, rafforzamenti od
indebolimenti, casi particolari o generalizzazioni.
In questo caso,
alcuni studiosi avevano
considerato il seguente problema
di carattere generale:
per quali insiemi di numeri interi \'e vero che se piazzo
numeri da questo insieme sugli archi del grafo,
soddisfando l'ovvia condizione che le somme debbano essere
pari tutt'intorno ad ogni nodo,
allora esiste sempre
una famiglia di circuiti tali che ogni arco
si trova coinvolto nel prescritto numero di circuiti?
In pratica, se l'insieme di interi $\{2\}$
avesse questa proprit\'a
la congettura seguirebbe. 
Era stato congetturato che
ogni insieme di interi non contenente l'uno
avesse questa propriet\'a.
Qui noi si \'e dato un controesempio
a questo rafforzamento evidentemente eccessivo.\\

  \vspace{1.4mm}
  \item[] {\sc Romeo Rizzi},
   \newblock  The Petersen graph is not the only $3$-edge
              connected superstrong snark,
   \newblock {\em ready for submission}\\
{\bf descrizione:}
Bill Jackson aveva osservato come
un controesempio minimale alla
{\em Double Cover Conjecture}
dovesse essere un
{\em $3$-edge connected superstrong snark}.
Ad oggi l'unico $3$-edge connected superstrong snark
conosciuto era il grafo di Petersen,
e per esso la congettura vale.
Qui noi forniamo una famiglia infinita di
$3$-edge connected superstrong snarks.\\

  \vspace{1.4mm}
  \item[] {\sc Romeo Rizzi},
   \newblock  Cycle cover property and $CPP=SCC$ property are not equivalent,
   \newblock {\em in preparation}\\
{\bf descrizione:}
Controesempio ad una congettura di Zhang.\\

  \vspace{1.4mm}
  \item[] {\sc Romeo Rizzi},
   \newblock  On $3$-connected graphs without even cycle decompositions,
   \newblock {\em submitted}\\
{\bf descrizione:}
Era stato congetturato che $K_5$,
il grafo completo su 5 nodi,
fosse l'unico grafo euleriano $3$-connesso
con un numero pari di archi a non poter essere
decomposto in circuiti di lunghezza pari.
Qui noi esibiamo una famiglia infinita
di controesempi.\\
\end{itemize}



\section{Grafi Perfetti}

\begin{itemize}
  \vspace{1.4mm}
  \item[] {\sc Romeo Rizzi},
   \newblock  On the Recognition of $P_4$-Indifferent Graphs,
   \newblock {\em submitted} \\
{\bf descrizione:}
Un grafo \'e
$P_4${\em -indifferent}
se ammette un ordinamento dei nodi
tale che per ogni cammino indotto
da quattro nodi $a,b,c,d$
si ha che $a<b<c<d$
oppure che $a>b>c>d$.
La famiglia dei grafi $P_4$-indifferent
include quella degli {\em indifferent}
ed inoltre ogni grafo che sia $P_4$-indifferent
\'e {\em perfectly orderable}.
L'interesse per grafi perfectly orderable
\'e motivato dall'osservazione di Chv\'atal
che un semplice algoritmo greedy
lungo l'ordinamento
produce sempre una colorazione ottima dei nodi.
Tuttavia il riconoscimento di grafi perfectly orderable
in generale \'e NP-completo.
Recentemente, Ho\`ang, Maffray e Noy
avevano dato una caratterizzazione dei grafi $P_4$-indifferent
in termini di sottografi indotti proibiti.
Noi riorganizziamo la loro dimostrazione
e ne deriviamo un algoritmo lineare per il riconoscimento
di grafi $P_4$-indifferent.
Se il grafo in input \'e $P_4$-indifferent,
allora l'algoritmo produce inoltre un ordinamento $<$
come le propriet\'a di cui sopra.\\
\end{itemize}



\section{Packing e Covering}

\begin{itemize}
  \vspace{1.4mm}
  \item[] {\sc Alberto Caprara, Alessandro Panconesi, Romeo Rizzi},
   \newblock  Packing triangles in planar graphs,
   \newblock {\em ready for submission}\\
{\bf descrizione:}
Dato un grafo planare,
vogliamo trovare in esso il massimo numero di triangoli disgiunti.
A seconda che li si voglia solamente disgiunti sui nodi od
anche sugli archi abbiamo due distinti problemi.
Tuttavia, in una trattazione assai unificata,
diamo dei PTAS per entrambi i problemi
avvalendoci del teorema separatore.
Un PTAS \'e un algoritmo approssimante
a qual si voglia precisione
ed \'e in pratica il meglio che si possa sperare
quando il problema in questione
risulti essere NP-hard.
In effetti noi si dimostra inoltre che entrambi i problemi sono
NP-hard.
Precedentemente solamente il caso di triangoli disgiunti
sui nodi era noto essere tale e la dimostrazione
basava su una ben pi\'u complessa riduzione
che non quella da noi proposta.\\

  \vspace{1.4mm}
  \item[] {\sc Alberto Caprara, Alessandro Panconesi, Romeo Rizzi},
   \newblock  Packing Cuts in Graphs,
   \newblock {\em in preparation}\\
{\bf descrizione:}
Diamo svariati risultati di inapprossimabilit\'a
per il problema di trovare il maggior numero possibile di
tagli disgiunti in un dato grafo.
La questione dell' NP-hardness del problema
era stata recentemente posta al centro dell'attenzione
da alcuni lavori di Ajeev,
ma segue ora come conseguenza di questi ben pi\'u forti risultati.
(Non approssimabilit\'a entro un fattore $\log n$ nel caso generale
ed APX-hardness in svariati casi particolari).
Diamo altres\'\i\ degli algoritmi approssimati che mostrano
come tali risultati siano stretti.\\

  \vspace{1.4mm}
  \item[] {\sc Richard C.~Brewster, Pavol Hell, Sarah H.~Pantel, Romeo Rizzi, Anders Yeo},
   \newblock  Packing paths in digraphs,
   \newblock {\em accepted by Journal of Graph Theory}\\
{\bf descrizione:}
Un matching pu\'o essere considerato
come una collezione di copie di $K_2$
disgiunte sui nodi.
Una generalizzazione del probema di trovare un matching
di massima cardinalit\'a \'e pertanto quella di,
fissata una famiglia di grafi $\G$,
e dato un certo grafo in input $G$,
trovare una collezione di sottografi di $G$ disgiunti sui nodi,
tutti isomorfi a qualche membro di $\G$.
Questo problema, noto come il $\G$-packing problem,
ha ricevuto molta attenzione nella letteratura.
Tuttavia poco era noto nel caso $\G$
sia una famiglia di grafi diretti,
se non che questa
cornice fosse pi\'u generale.
In questo lavoro si \'e dato inizio all'esplorazione
del caso in cui $\G$
\'e una famiglia di grafi diretti.
In particolare, si \'e affrontato
e completamente caratterizzato
il caso in cui $\G$ \'e una famiglia di cammini diretti.
Il problema di coprire il maggior numero di nodi di
$G$ \'e risolvibile in tempo
polinomiale se $\G$ \'e riducibile
o alla famiglia $\G=\{\vec{P}_1\}$
o alla famiglia $\G=\{\vec{P}_1,\vec{P}_2\}$.
In tutti gli altri casi il
problema risulta essere $\NP$-completo.\\

  \vspace{1.4mm}
  \item[] {\sc Romeo Rizzi, Andr\'{a}s Seb\H{o}},
   \newblock A note on Matroid Flows,
   \newblock {\em in preparation}\\
{\bf descrizione:}
Il lavoro comincia con l'osservare
come una per altro importante
congettura di Seymour riguardante
una certa classe di matroidi
fosse di fatto mal posta.
Infatti,
ove la congettura fosse presa un po' troppo alla lettera,
diamo un controesempio alla stessa.
Forniamo quindi la versione rivista della congettura
e ne esploriamo le relazioni con altre
congetture e risultati.\\

  \vspace{1.4mm}
  \item[] {\sc Romeo Rizzi},
   \newblock  On packing $T$-joins,
   \newblock {\em in preparation}\\
{\bf descrizione:}
Investighiamo il legame tra
il problema della fattorizzazione di $r$-grafi
ed il problema dell'impaccamento di $T$-joins.
\end{itemize}



\section{Colorazione di archi}

\begin{itemize}
  \vspace{1.4mm}
  \item[] {\bf tesi di dottorato,}
          relatore: {\sc Prof.~Michele Conforti}
          (Dipartimento di Matematica, Universit\'a di Padova),
          controrelatore: {\sc Prof.~Bert Gerards}
          (Istituto di Ricerca CWI, Amsterdam).
          Innanzittutto devo segnalare che la
          mia tesi di dottorato aveva toccato questo argomento
          fornendo dei contributi originali.
          Il materiale in essa contenuto e riguardante
          la {\em Colorazione di archi} ha trovato
          spazio in alcuni dei seguenti lavori.\\

  \vspace{1.4mm}
  \item[] {\sc Ajai Kapoor, Romeo Rizzi},
   \newblock  Edge-coloring bipartite graphs,
   \newblock {\em accepted by Journal of Algorithms}\\
{\bf descrizione:}
Battiamo (o meglio battevamo)
il record asintotico
sulla complessit\'a computazionale
per produrre una soluzione ottima al
problema di colorare gli archi di un grafo
bipartito col minor numero possibile di colori.
Inoltre davamo un algoritmo di complessit\'a lineare
(e meglio di cos\'\i\ non \'e ovviamente possibile
per una macchina sequenziale)
per produrre una soluzione
che impieghi solamente un colore in pi\'u
che non la soluzione ottima.\\

  \vspace{1.4mm}
  \item[] {\sc Romeo Rizzi},
   \newblock  Finding $1$-factors in bipartite regular graphs,
              and edge-coloring bipartite graphs,
   \newblock {\em submitted} \\
{\bf descrizione:}
Questo lavoro migliora ulteriormente
lo stato del problema di colorare gli archi in un grafo
bipartito con al pi\'u $\Delta$ colori.
Il collo di bottiglia per la complessit\'a
computazionele dell'algoritmo proposto
da Kapoor
e Rizzi in ``Edge-coloring bipartite graphs''
era la procedura presa in prestito da Hopcroft e Cole
per il reperimento di un accoppiamento perfetto
in un grafo bipartito e regolare.
Qui si migliora su questa procedura
grazie a degli argomenti algebrici.
In lavoro mette in luce il legame tra
la fattorizzazione di grafi e la fattorizzazione
di numeri interi.\\

  \vspace{1.4mm}
  \item[] {\sc Alberto Caprara, Romeo Rizzi},  
   \newblock  Improving a Family of Approximation
              Algorithms to Edge Color Multigraphs,
   \newblock {\em Inf. Proc. Lett.} 68 (1998) 11-15.\\
{\bf descrizione:}
Coloriamo gli archi di un grafo
in modo che archi incidenti ad uno stesso nodo
abbiano colori diversi.
Evidentemente non possiamo sperare di impiegare
meno di $\Delta$ colori,
ove $\Delta$ \'e il massimo grado di un nodo.
Il teorema di Vizing dice che
possiamo sempre farcela
con $\Delta+1$ colori.
Tuttavia ci\'o non \'e pi\'u vero
ove si consideri la possibilit\'a di avere archi paralleli
tra una stessa copia di nodi.
\'E stato tuttavia definito
un parametro $\Delta'$,
di facile computazione,
che costituisce una miglior stima inferiore
che non $\Delta$ sul numero di colori necessari.
Delle famigerate congetture
degli anni 80
affermano poi che ce la si pu\'o sempre
fare con al pi\'u $\Delta'+1$ colori.
In ogni caso,
il problema di colorare
gli archi con pochi colori ha importanti applicazioni.
Questo spiega la serie di algoritmi che sono stati
via via presentati per impiegare al pi\'u
$\alpha \Delta'$ colori.
Purtroppo, ogni nuovo algoritmo se ne esce
con un valore di $\alpha$ pi\'u vicino ad 1,
ma anche con una complessit\'a computazionale
e di implementazione sempre maggiore.
Noi abbiamo mostrato come con un semplice trucco fosse possibile
migliorare il grado di approssimazione per ciascuno
degli algoritmi sinora proposti senza incrementarne
la complessit\'a.
Ci\'o consente di dedurre la validit\'a della congettura
di cui sopra per $\Delta\leq 12$.\\

  \vspace{1.4mm}
  \item[] {\sc Romeo Rizzi},
   \newblock  K\H{o}nig's Edge Coloring Theorem without augmenting paths,
   \newblock {\it Journal of Graph Theory} 29 (1998) 87.\\
{\bf descrizione:}
Il teorema dei balli \'e uno di
quei risultati con cui K\H{o}nig
ha posto le basi della teoria dei grafi per come oggi \'e conosciuta.
Il teorema ha applicazioni dirette in problemi di scheduling e timetabling,
ma anche in discipline pi\'u lontane come la statistica.
Nonostante la sua importanza, il risultato veniva derivato
come corollario di un'altro fondamentale teorema di K\H{o}nig.
Noi introduciamo  un'operazione elementare che
trasforma un grafo bipartito e $\Delta$-regolare
in un suo fratellino minore.
La dimostrazione \'e ora per induzione
ed \'e diretta.\\

\vspace{1.4mm}
  \item[] {\sc Romeo Rizzi},
   \newblock  Impaccando $T$-tagli e $T$-giunti,
   \newblock {\em Bollettino Sezione B dell'Unione Matematica Italiana},
   \newblock {Fascicolo speciale dedicato alle tesi di dottorato}
             (8) 1-A Suppl. (1998) 201-204.\\
{\bf descrizione:}
L'UMI offriva uno spazio di 4 pagine per
presentare la tesi di dottorato nel suo prestigioso bollettino.
In italiano, e con delle figure,
ho cercato di attirare l'attenzione del lettore
introducendo solo un paio dei problemi affrontati
nella mia tesi.
Risultato? Un successo:
qualcuno ha risposto
chiedendomi copia della tesi.\\
\end{itemize}



\section{Fattorizzazione di grafi}

\begin{itemize}
  \vspace{1.4mm}
  \item[] {\bf tesi di dottorato,}
          relatore: {\sc Prof.~Michele Conforti}
          (Dipartimento di Matematica, Universit\'a di Padova),
          controrelatore: {\sc Prof.~Bert Gerards}
          (Istituto di Ricerca CWI, Amsterdam).
          Innanzittutto devo segnalare che la
          mia tesi di dottorato aveva toccato questo argomento
          fornendo dei contributi originali.
          Il materiale in essa contenuto e riguardante
          la {\em Fattorizzazione di grafi} ha in gran parte trovato
          spazio in alcuni dei seguenti lavori.\\

  \vspace{1.4mm}
  \item[] {\sc Romeo Rizzi},
   \newblock  Indecomposable $r$-graphs and some other counterexamples,
   \newblock {\it Journal of Graph Theory} 32 (1) (1999) 1--15.\\
{\bf descrizione:}
Un $r$-grafo
\'e un punto intero nel cono
generato proiettando il politopo dei matching
perfetti in un grafo completo.
Noi diamo dei controesempi a delle congetture
di Seymour e di altri ricercatori
e mostriamo come le altezze per le basi
di Hilbert di questi coni non siano limitate
da costante alcuna.
Paul Seymour in persona mi ha chiesto
di sottomettere questo lavoro al suo giornale
spendendo significativi apprezzamenti.\\

  \vspace{1.4mm}
  \item[] {\sc Romeo Rizzi},
   \newblock  Finding $1$-factors in bipartite regular graphs,
              and edge-coloring bipartite graphs,
   \newblock {\em submitted} \\
{\bf descrizione:}
Questo lavoro migliora ulteriormente
lo stato del problema di colorare gli archi in un grafo
bipartito con al pi\'u $\Delta$ colori.
Il collo di bottiglia per la complessit\'a
computazionele dell'algoritmo proposto
da Kapoor
e Rizzi in ``Edge-coloring bipartite graphs''
era la procedura presa in prestito da Hopcroft e Cole
per il reperimento di un accoppiamento perfetto
in un grafo bipartito e regolare.
Qui si migliora su questa procedura
grazie a degli argomenti algebrici.
In lavoro mette in luce il legame tra
la fattorizzazione di grafi e la fattorizzazione
di numeri interi.\\

\vspace{1.4mm}
  \item[] {\sc Romeo Rizzi},
   \newblock  Impaccando $T$-tagli e $T$-giunti,
   \newblock {\em Bollettino Sezione B dell'Unione Matematica Italiana},
   \newblock {Fascicolo speciale dedicato alle tesi di dottorato}
             (8) 1-A Suppl. (1998) 201-204.\\
{\bf descrizione:}
L'UMI offriva uno spazio di 4 pagine per
presentare la tesi di dottorato nel suo prestigioso bollettino.
In italiano, e con delle figure,
ho cercato di attirare l'attenzione del lettore
introducendo solo un paio dei problemi affrontati
nella mia tesi.
Risultato? Un successo:
qualcuno ha risposto
chiedendomi copia della tesi.\\

  \vspace{1.4mm}
  \item[] {\sc Romeo Rizzi},
   \newblock  K\H{o}nig's Edge Coloring Theorem without augmenting paths,
   \newblock {\it Journal of Graph Theory} 29 (1998) 87.\\
{\bf descrizione:}
Il teorema dei balli \'e uno di
quei risultati con cui K\H{o}nig
ha posto le basi della teoria dei grafi per come oggi \'e conosciuta.
Il teorema ha applicazioni dirette in problemi di scheduling e timetabling,
ma anche in discipline pi\'u lontane come la statistica.
Nonostante la sua importanza, il risultato veniva derivato
come corollario di un'altro fondamentale teorema di K\H{o}nig.
Noi introduciamo  un'operazione elementare che
trasforma un grafo bipartito e $\Delta$-regolare
in un suo fratellino minore.
La dimostrazione \'e ora per induzione
ed \'e diretta.\\

  \vspace{1.4mm}
  \item[] {\sc Romeo Rizzi},
   \newblock  On packing $T$-joins,
   \newblock {\em in preparation}\\
{\bf descrizione:}
Investighiamo il legame tra
il problema della fattorizzazione di $r$-grafi
ed il problema dell'impaccamento di $T$-joins.
\end{itemize}



\section{Problemi di cammini minimi}

\begin{itemize}
  \vspace{1.4mm}
  \item[] {\sc Michele Conforti, Romeo Rizzi},  
   \newblock  Shortest Paths in Conservative Graphs,
   \newblock {\em accepted by Discrete Mathematics}\\
{\bf descrizione:}
L'algoritmo di Ford e Fulkerson
computa i cammini minimi in un grafo orientato
con pesi razionali sugli archi,
accertando al contempo l'assenza di cicli di peso totale
negativo.
Ma quando il grafo non \'e orientato,
nessun algoritmo specifico era noto per lo stesso problema,
che doveva venir risolto con tecniche di matching.
In effetti queste tecniche mostravano come
il problema fosse di fatto pi\'u difficile che non nel caso diretto.
In questo lavoro abbiamo definito due operazioni elementari
che, combinate, riescono a ridurre il grafo dato in input
ad un nodo singolo
a meno che un'{\em anomalia} non venga riscontrata
in una certa fase del processo di riduzione.
Se il grafo si riduce ad un nodo singolo
allora ne deduciamo che esso non conteneva
cicli negativi ed i cammini correntemente impugnati sono ottimi.
Se un'anomalia \'e riscontrata,
allora, procedendo a ritroso,
o scopriamo di poter migliorare uno dei cammini correntemente considerati,
o reperiamo un ciclo negativo.
Ci\'o ci consente di ottenere
un algoritmo specifico
per il problema dei cammini minimi su grafi non diretti
e conservativi (= senza cicli negativi).
L'algoritmo pu\'o essere
considerato come un algoritmo alternativo
per problemi di matching in generale.
Inoltre tramite esso \'e possibile
fornire dimostrazione algoritmica
di varie propriet\'a strutturali
di grafi conservativi e cammini minimi.\\ 
\end{itemize}



\section{Problemi di taglio minimo}

\begin{itemize}
  \vspace{1.4mm}
  \item[] {\sc Romeo Rizzi},
   \newblock  On minimizing symmetric set functions,
   \newblock {\em accepted by Combinatorica}\\
{\bf descrizione:}
Talvota una dimostrazione semplice \'e la base
di una nuova generalizzazione.
\'E il caso di questo lavoro dove si \'e data
una dimostrazione del tutto elementare
di un lemma un po' magico che legava
certi ordinamenti dei nodi di un grafo
con delle coppie di nodi che erano buone,
nel senso che consentivano di ridurre
il problema del taglio minimo.
Il metodo era stato generalizzato precedentemente da
Queyranne per la minimizzazione
di funzioni simmetriche e submodulari,
ma i suoi argomenti erano complessi e poco trasparenti.
Noi si \'e data una dimostrazione induttiva
talmente elementare da non richiedere
quasi alcuna ipotesi.
Ci\'o ha portato a definire una classe di funzioni
sulla base delle sole propriet\'a utilizzate
ed a derivare poi le varie generalizzazioni precedentemente
conosciute come casi particolari
della nostra analisi.
Il tutto in grande semplicit\'a.
Questo lavoro mi ha meritato l'apprezzamento da parte
dei pi\'u insigni studiosi nel settore.\\

  \vspace{1.4mm}
  \item[] {\sc Romeo Rizzi},
   \newblock  Excluding a simple good pair approach to directed cuts,
   \newblock {\em accepted by Graphs and Combinatorics}\\
{\bf descrizione:}
Un controesempio non particolarmente difficile ma tecnicamente
necessario.
Recentemente, un nuovo ed efficace approccio
era stato proposto per
il problema del taglio minimo
su grafi non-diretti.
Volevamo assicurarci che lo stesso approccio
non poteva essere esteso in quanto tale al caso di grafi diretti.\\

  \vspace{1.4mm}
  \item[] {\sc Romeo Rizzi},
   \newblock  Efficient implementations of greedy order algorithms,
   \newblock {\em submitted} \\
{\bf descrizione:}
Nel caso del problema del taglio minimo,
Nagamochi ed Ibaraki avevano mostrato come
alcuni accorgimenti fossero effettivi nel migliorare
le performances dell'algoritmo da loro stessi proposto
per il problema.
Questa analisi non era stata possibile per
le successive generalizzazioni del loro approccio causa
l'eccessiva complessit\'a delle stesse.
A seguito del nostro successo
in ``On minimizing symmetric set functions'',
tale strada diviene per\'o percorribile.
Qui noi mostriamo come gli accorgimenti di Nagamochi ed Ibaraki
possano essere di fatto inclusi
nell'impostazione assiomatica da noi proposta
nel lavoro di cui sopra.
Di fatto il nostro approccio
consente persino di perfezionare da un lato,
e di semplificare dall'altro,
gli accorgimenti in questione,
Tali perfezionamenti si applicano
anche nel caso particolare ed originale
del problema del taglio minimo.\\

  \vspace{1.4mm}
  \item[] {\sc Romeo Rizzi},
   \newblock  A new method for finding cuts of minimum weight,
   \newblock {\em in preparation}\\
{\bf descrizione:}
Forniamo un metodo nuovo
ed invero semplice per il problema del taglio minimo
in grafi.
Il metodo generalizza al problema del taglio minimo in ipergrafi.
Inoltre l'idea base promette bene
in quanto ad applicabilit\'a ad altri problemi di edge-connectivity.
\end{itemize}



\section{Biologia Computazionale}

\begin{itemize}
  \vspace{1.4mm}
  \item[] {\sc Alberto Caprara, Romeo Rizzi},
   \newblock  Improved Breakpoint Graph Decompositions,
   \newblock {\em submitted}\\
{\bf descrizione:}
Il sorting by reversals \'e uno di quei
problemi in biologia computazionale
che sono venuti alla
luce con lo sviluppo del Progetto Genoma Umano.
L'interesse per questo problema da parte
della biologia computazionale \'e forte.
Caprara aveva purtroppo dimostrato che il problema
era NP-hard.
Sorge a questo punto l'interesse per degli algoritmi
di tipo approssimato.
Questi sono intesi a produrre delle soluzioni che, se non ottime,
siano quantomeno di provata qualit\'a,
ossia di qualit\'a superiore ad un certo bound fissato.
Noi si \'e migliorato il miglior
bound attuale.
Il tutto a prezzo di un'analisi combinatorica
di non poco peso per il lettore e per chi mai
volesse implementare l'algoritmo.
La complessit\'a computazionale dell'algoritmo
tuttavia non ne ha sofferto.\\
\end{itemize}



\vspace{1.6cm}
Amsterdam,
November 1999.\\
\vspace{2mm}

Romeo Rizzi

\end{document}












